\documentclass[times,10pt]{sigplanconf}

% Remote packages

% For pdflatex, replaced by fontspec:
% \usepackage{tgpagella}
\usepackage[T1]{fontenc}
\usepackage[utf8]{inputenc}

% For xelatex or lualatex:
% \usepackage{fontspec}
% \setmainfont{Times New Roman}


\usepackage{amsmath}
\usepackage{amsthm}
\usepackage{amssymb}
\usepackage{mathtools} % For \Coloneqq
\usepackage{bm}        % Bold symbols in maths mode
\usepackage{cmll}
\usepackage{fixltx2e}
\usepackage{stmaryrd}
\usepackage[dvipsnames]{xcolor}
\usepackage{listings} % For code listings
% \usepackage{minted}
% \usemintedstyle{murphy}
\usepackage{fancyvrb}
\usepackage{url}
\usepackage{xspace}
\usepackage{comment}
\usepackage{mdwlist}

% Typography
\usepackage[euler-digits,euler-hat-accent]{eulervm}

% Copied from the FCore paper:
\usepackage[colorlinks=true,allcolors=black,breaklinks,draft=false]{hyperref}   % hyperlinks, including DOIs and URLs in bibliography
% known bug: http://tex.stackexchange.com/questions/1522/pdfendlink-ended-up-in-different-nesting-level-than-pdfstartlink

% Figures with borders
% http://en.wikibooks.org/wiki/LaTeX/Floats,_Figures_and_Captions
% \usepackage{float}
% \floatstyle{boxed}
% \restylefloat{figure}

% Local packages

% \usepackage{styles/bcprules}    % by Benjamin C. Pierce
\usepackage{styles/mathpartir}  % by Didier Rémy


% ! Always load mathastext last
% http://mirrors.ibiblio.org/CTAN/macros/latex/contrib/mathastext/mathastext.pdf
% \renewcommand\familydefault\ttdefault
% \usepackage{mathastext}
% \renewcommand\familydefault\rmdefault

% http://tex.stackexchange.com/questions/114151/how-do-i-reference-in-appendix-a-theorem-given-in-the-body
\usepackage{thmtools, thm-restate}
\newtheorem{theorem}{Theorem}
\newtheorem{lemma}{Lemma}


\begin{document}

\special{papersize=8.5in,11in}
\setlength{\pdfpageheight}{\paperheight}
\setlength{\pdfpagewidth}{\paperwidth}

\conferenceinfo{CONF 'yy}{Month d--d, 20yy, City, ST, Country}
\copyrightyear{20yy}
\copyrightdata{978-1-nnnn-nnnn-n/yy/mm}
\doi{nnnnnnn.nnnnnnn}

\titlebanner{banner above paper title}        % These are ignored unless
\preprintfooter{\name}                        % 'preprint' option specified.

\title{System \name: A Simple Core Language for Extensibility}
%%\subtitle{Subtitle Text, if any}

\authorinfo{Name1}
           {Affiliation1}
           {Email1}
\authorinfo{Name2\and Name3}
           {Affiliation2/3}
           {Email2/3}

\maketitle

\begin{abstract}
  
  Over the years there have been various proposals for \emph{design
    patterns} to improve \emph{extensibility} of programs.
  Examples include \emph{Object Algebras}, \emph{Modular Visitors} or
  Torgersen's design patterns using generics.
  Although those design patterns give practical
  benefits in terms of extensibility, they also expose limitations in
  existing mainstream OOP languages. Some pressing
  limitations are: 1) lack of good mechanisms for
  \emph{object-level} composition; 2) \emph{conflation of 
    (type) inheritance with subtyping}; 3) \emph{heavy reliance on generics}.

  This paper presents System \name: an extension of System F with
  \emph{intersection types} and a \emph{merge operator}.  The goal of System \name
  is to study the minimal language constructs needed to support
  various extensible designs, while at the same time addressing the
  limitations of existing OOP languages. To address the lack of good
  object-level composition mechanisms, System \name uses the merge
  operator to do dynamic composition of values/objects. Moreover, in
  System \name type inheritance is independent of subtyping, and an
  extension can be a supertype of a base object type.  Finally, System
  \name replaces many uses of generics by intersection types or
  conventional subtyping. System \name is formalized and
  implemented. Moreover the paper shows how various extensible designs
  can be encoded in System \name.

\end{abstract}

%%\category{CR-number}{subcategory}{third-level}

% general terms are not compulsory anymore,
% you may leave them out
%%\terms
%%Design, Languages, Theory

%%\keywords
%%Intersecion Types, Polymorphism, Type System

\section{Introduction}

There has been a remarkable number of works aimed at improving support
for extensibility in programming languages. The motivation behind this
line of work is simple, and it is captured quite elegantly by the
infamous \emph{Expression Problem~}\cite{wadler1998expression}: there
are \emph{two} common and desirable forms of extensibility, but most
mainstream languages can only support one type well. Unfortunately
the lack of support in the other direction has significant
consequences in terms of code maintence and software evolution.  As a
result researchers proposed various approaches to address the problem,
including: visions of new programming
models~\cite{Prehofer97,Tarr99ndegrees,Harrison93subject}; new
programming languages or language
extensions~\cite{McDirmid01Jiazzi,Aracic06CaesarJ,Smaragdakis98mixin,nystrom2006j},
and \emph{design patterns} that can be used with existing mainstream
languages~\cite{togersen:2004,Zenger-Odersky2005,oliveira09modular,oliveira2012extensibility}.

Some of the more recent work on extensibility is focused on design
patterns. Examples include \emph{Object
  Algebras}~\cite{oliveira2012extensibility}, \emph{Modular Visitors}~\cite{oliveira09modular,togersen:2004} or
Torgersen's~\cite{togersen:2004} four design patterns using generics. In those
approaches the idea is to use some advanced (but already available)
features, such as \emph{generics}~\cite{Bracha98making}, in combination with conventional
OOP features to model more extensible designs.  Those designs work in
modern OOP languages such as Java, C\# or Scala.

Although such design patterns give practical benefits in terms of
extensibility, they also expose limitations in existing mainstream OOP
languages. In particular there are three pressing limitations: 
1) lack of good mechanisms for
  \emph{object-level} composition; 2) \emph{conflation of 
    (type) inheritance with subtyping}; 3) \emph{heavy reliance on generics}.

  The first limitation shows up, for example, encodings of Feature-Oriented
  Programming~\cite{Prehofer97} or Attribute Grammars~\cite{Knuth1968} using Object
  Algebras~\cite{oliveira2013feature,rendel14attributes}. These programs are best
  expressed using a form of \emph{type-safe}, \emph{dynamic},
  \emph{delegation}-based composition. Although such form of
  composition can be encoded in languages like Scala, it requires the
  use of low-level reflection techniques, such as dynamic proxies,
  reflection or other forms of meta-programming. It is clear
  that better language support would be desirable.

  The second limitation shows up in designs for modelling
  modular or extensible visitors~\cite{togersen:2004,oliveira09modular}.  The vast majority of modern
  OOP languages combines type inheritance and subtyping. 
  That is a type extension induces a subtype. However
  as Cook et al.~\cite{cook1989inheritance} famously argued there are programs where
  ``\emph{subtyping is not inheritance}''. Interestingly 
  not many programs have been previously reported in the literature
  where the distinction between subtyping and inheritance is
  relevant in practice. However, as shown in this paper, it turns out that this
  difference does show up in practice when designing modular
  (extensible) visitors.  We believe that modular visitors provide a
  compeling example where inheritance and subtyping should
  not be conflated!

  Finally, the third limitation is prevalent in many extensible
  designs~\cite{togersen:2004,Zenger-Odersky2005,oliveira09modular,oliveira2013feature,rendel14attributes}.
  Such designs rely on advanced features of generics,
  such as \emph{f-bounded polymorphism}~\cite{Canning89f-bounded}, \emph{variance
    annotations}~\cite{Igarashi06variant}, \emph{wildcards}~\cite{Torgersen04wildcards} and/or \emph{higher-kinded
    types}~\cite{Moors08generics} to achieve type-safety. Sadly, the amount of
  type-annotations, combined with the lack of understanding of these
  features, usually deters programmers from using such designs.

\begin{comment}
Motivated by the insights gained in previous work, this paper presents 
a minimal core calculus that addresses current limitations and
provides a better foundational model for statically typed
delegation-based OOP? We show that Object Algebras fit nicely in this
model. 
\end{comment}

This paper presents System \name (pronounced \emph{f-and}): an extension of System F~\cite{Reynolds74f}
with intersection types and a merge operator~\cite{dunfield2014elaborating}.  The goal of
System \name is to study the \emph{minimal} foundational language
constructs that are needed to support various extensible designs,
while at the same time addressing the limitations of existing OOP
languages. To address the lack of good object-level composition
mechanisms, System \name uses the merge operator for dynamic
composition of values/objects. Moreover, in System \name (type-level)
extension is independent of subtyping, and it is possible for an
extension to be a supertype of a base object type. Furthermore,
intersection types and conventional subtyping can be used in many
cases instead of advanced features of generics. Indeed this paper 
shows how many previous designs in the literature can be encoded 
without such advanced features of generics.


Technically speaking System \name is mainly inspired by the work of
Dundfield~\cite{dunfield2014elaborating}.  Dundfield shows how to model a simply typed
calculus with intersection types and a merge operator. The presence of
a merge operator adds significant expressiveness to the language,
allowing encodings for many other language constructs as syntactic
sugar. System \name differs from Dundfield's work in a few
ways. Firstly it adds parametric polymorphism and formalizes an
extension for records to support a basic form of objects. Secondly,
the elaboration semantics into System F is done directly from the
source calculus with subtyping. In contrast Dunfield has an additional
step which eliminates subtyping.  Finally a non-technical difference
is that System \name is aimed at studying issues of OOP languages and
extensibility, whereas Dunfield's work was aimed at Functional
Programming and he did not consider applications to extensibility.
Like many other foundational formal models for OOP (for
example $F_{<:}$~\cite{CMMS}), System \name is purely functional and it uses
structural typing.

%%System \name is
%%formalized and implemented. Furthermore the paper illustrates how
%%various extensible designs can be encoded in System \name.

\begin{comment}
We present a polymorphic calculus containing intersection types and records, and show
how this language can be used to solve various common tasks in functional
programming in a nicer way.Intersection types provides a power mechanism for functional programming, in
particular for extensibility and allowing new forms of composition.

Prototype-based programming is one of the two major styles of object-oriented
programming, the other being class-based programming which is featured in
languages such as Java and C\#. It has gained increasing popularity recently
with the prominence of JavaScript in web applications. Prototype-based
programming supports highly dynamic behaviors at run time that are not possible
with traditional class-based programming. However, despite its flexibility,
prototype-based programming is often criticized over concerns of correctness and
safety. Furthermore, almost all prototype-based systems rely on the fact that
the language is dynamically typed and interpreted.
\end{comment}

In summary, the contributions of this paper are:

\begin{comment}
\george{Typing extensible records with a minimalistic design might also be of
  interest. We manage to achieve all what Daan Leijen achieved in
  \url{http://research.microsoft.com/pubs/65409/scopedlabels.pdf} except field
  renaming, with simpler approach.}
\end{comment}

\begin{itemize}

\item {\bf A Minimal Core Language for Extensibility:} This paper
  identifies a minimal core language, System \name, capable of
  expressing various extensibility designs in the literature.
  System \name also addresses limitations of existing OOP
  languages that complicate extensible designs. 
  
\item {\bf Formalization of System \name:} An elaboration semantics of
  System \name into System F is given, and type-soundness is proved.

\item {\bf Encodings of Extensible Designs:} Various encodings of
  extensible designs into System \name, including \emph{Object
    Algebras} and \emph{Modular Visitors}. 

\item {\bf A Practical Example where ``Inheritance is not Subtyping''
    Matters:} This paper shows that in modular/extensible visitors
  suffer from the ``inheritance is not subtyping problem''. 
%% Moreover 
%% with extensible visitors the extension should become a
%% \emph{supertype}, not a subtype. \bruno{extension with accept method}

\item {\bf Implementation and Examples:} An implementation of an
  extension of System \name, as well as the examples presented in the
  paper, are publicly available\footnote{{\bf Note to reviewers:} Due
    to the anonymous submission process, the code (and some machine
    checked proofs) is submitted as supplementary material.}. 

\begin{comment}

\item{elaboration typing rules which given a source expression with intersection
    types, typecheck and translate it into an ordinary F term. Prove a type
    preservation result: if a term $ e $ has type $ \tau $ in the source language,
    then the translated term $ \im e $ is well-typed and has type $ \im \tau $ in the
    target language.}

\item{present an algorithm for detecting incoherence which can be very important
    in practice.}

\item{explores the connection between intersection types and object algebra by
    showing various examples of encoding object algebra with intersection
    types.}

\end{comment}

\end{itemize}

\begin{comment}
\subsection{Other Notes}

finitary overloading: yes
but have other merits of intersection been explored?

-- Compare Scala:
-- merge[A,B] = new A with B

-- type IEval  = { eval :  Int }
-- type IPrint = { print : String }

-- F[\_]
\end{comment}
\input{sections/overview.tex}
\section{Applications to Extensibility} \label{sec:application}

% Structural subtyping facilitates reuse~\cite{malayeri2008integrating}.
% \bruno{orphan sentence!}

\bruno{Make sure that the important code in the paper is reused from a script
  and not inlined directly in the text.}

This section shows that, although \name is a minimal language, its
features are enough for encoding extensible designs that been
presented in mainstream languages. Moreover \name addresses
limitations of those languages, making those designs significantly
simpler. There are two main advantages of \name over existing
languages:

\begin{enumerate}
\item \name supports dynamic composition of intersecting values.
\item \name does not couple type inheritance and subtyping. Moreover
  \name supports contravariant parameter types in the subtyping relation.
\end{enumerate}

These two features avoid the use of low-level programming techniques,
and make the designs less reliant on advanced features of generics. 

\begin{comment}
Various solutions have been proposed to deal with the extensibility problems and
many rely on heavyweight language features such as abstract methods and classes
in Java. These two features can be used to improve existing designs of modular
programs.

\bruno{I would like to see a story about Church Encodings in
  \name. Can you look at Pierce's papers and try to write something
  along those lines? That will be a good intro for object algebras and
visitors!}


\url{http://repository.cmu.edu/cgi/viewcontent.cgi?article=3018&context=compsci}
Church encoding allows modelling algebraic data types. 
\end{comment}

% Introduce the expression problem



% There has been recently a lightweight solution to the expression problem that
% takes advantage of covariant return types in Java. We show that \name is able
% to solve the expression problem in the same spirit.

% - Object/Fold Algebras. How to support extensibility in an easier way.

% See Datatypes a la Carte

% - Mixins

% - Lenses? Can intersection types help with lenses? Perhaps making the
% types more natural and easy to understand/use?

% - Embedded DSLs? Extensibility in DSLs? Composing multiple DSL interpretations?

% http://www.cs.ox.ac.uk/jeremy.gibbons/publications/embedding.pdf

% \bruno{You already talk about overloading in the previous section. Need to
% decide where to put the text!}

% Dunfield~\cite{dunfield2014elaborating} notes that using merges as a mechanism
% of overloading is not as powerful as type classes.

% Multiple inheritance?
% Algebra -> P1,2
% Visitor -> P2

% Yanlin
% Mixin

% \begin{lstlisting}
% let merge A B (f : ExpAlg A) (g : ExpAlg B) = {
%   lit = \(x : Int). f.lit x ,, g.lit x,
%   add = \(x : A & B). \(y : A & B). f.add x y ,, g.add x y
% };
% \end{lstlisting}

\subsection{Object Algebras}\label{subsec:OAs}

% Object algebras provide an alternative to \emph{algebraic data types}
% (ADT).\bruno{We are targeting an OO crowd. Mentioning algebraic
%   datatypes is not going to be very useful there.}

%  For example, the
% following Haskell definition of the type of simple expressions
% \begin{lstlisting}{language=haskell}
% data Exp where
%   Lit :: Int -> Exp
%   Add :: Exp -> Exp -> Exp
% \end{lstlisting}
% can be expressed by the \emph{interface} of an object algebra of
% simple expressions:
% \begin{lstlisting}{language=scala}
% trait ExpAlg[E] {
%   def lit(x: Int): E
%   def add(e1: E, e2: E): E
% }
% \end{lstlisting}
% Similar to ADT, data constructors in object algebras are represented by functions such as
% \lstinline{lit} and \lstinline{add} inside an interface \lstinline{ExpAlg}.
% Different with ADT, the type of the expression itself is abtracted by a type
% parameter \lstinline{E}.

% which can be expressed similarly in \name as:
% \begin{lstlisting}{language=F2J}
% type ExpAlg E = {
%   lit : Int -> E,
%   add : E -> E -> E
% }
% \end{lstlisting}

% Introduce Scala's intersection types

% Scala supports intersection types via the \lstinline{with} keyword. The type
% \lstinline{A with B} expresses the combined interface of \lstinline{A} and
% \lstinline{B}. The idea is similar to
% \begin{lstlisting}{language=java}
% interface AwithB extends A, B {}
% \end{lstlisting}
% in Java.
% \footnote{However, Java would require the \lstinline{A} and \lstinline{B} to be
%   concrete types, whereas in Scala, there is no such restriction.}

% The value level counterpart are functions of the type \lstinline
% {A => B => A with B}. \footnote{FIXME}

Oliveira and Cook~\cite{oliveira2012extensibility} proposed a design pattern that can solve the
Expression Problem in languages like Java. An advantage of the pattern
over previous solutions is that it is relatively lightweight in terms
of type system features. In a latter paper, Oliveira et al.~\cite{oliveira2013feature}
noted some limitations of the original design pattern and proposed 
some new techniques that generalized the original pattern, allowing it 
to express programs in a Feature-Oriented Programming~\cite{Prehofer97} style.
Key to these techniques was the ability to dynamically compose object
algebras.

Unfortunatelly, dynamic composition of object algebras is
non-trivial. At the type-level it is possible to express the resulting
type of the composition using intersection types. Thus, it is still
possible to solve that part problem nicely in a language like Scala (which
has basic support for intersection types). However, the dynamic
composition itself cannot be easily encoded in Scala. The fundamental 
issue is that Scala lacks a \lstinline{merge} operator (see the
discussion in Section~\ref{subsec:interScala}). Although both Oliveira et al.~\cite{oliveira2013feature} and
Rendell et al.~\cite{rendel14attributes} have shown that such a \lstinline{merge} operator can
be encoded in Scala, the encoding fundamentally relies in low-level
programming techniques such as dynamic proxies, reflection or
meta-programming. 

Because \name supports a \lstinline{merge} operator natively, dynamic
object algebra composition becomes easy to encode. The remainder of
this section shows how object algebras and object algebras composition
can be encoded in \name. We will illustrate this point with an
step-by-step of solving the Expression Problem. 
%%Prior knowledge of object algebras is not assumed.
 
% can be cumbersome and
% language support for intersection types would solve that problem. 
% Our type system is just a simple extension of System $ F $; yet surprisingly, it
% is able to solve the limitations of using object algebras in languages such as
% Java and Scala.

\paragraph{A simple system of arithmetic expressions.} 
In the Expression Problem, the idea is to start with a very simple
system modeling arithmetic expressions and evaluation.
The initial system considers expressions with two variants (literals and
addition) and one operation (evaluation). Here is an interface that supports
evaluation:
\begin{lstlisting}{language=F2J}
type IEval = {eval: Int};
\end{lstlisting}

\noindent In \name the interfaces of objects (or object types) are expressed as
a record type. A \lstinline{type} declaration allow us to create a
simple alias for a type.  In this case \lstinline{IEval} is an alias
for \{\lstinline{eval: Int}\}.

With object algebras, the idea is to create an object algebra
interface, \lstinline$ExpAlg$, for expression types with the two
variants. This interface has a fixed number of variants, but abstracts over the
the type of the interpretation \lstinline$E$.

\begin{lstlisting}{language=F2J}
type ExpAlg[E] = {
   lit: Int -> E, 
   add: E -> E -> E
};
\end{lstlisting}
% whereas
%\lstinline$ExpAlg[IEval & IPrint]$ will be the type of object algebras that
%support both evaluation and pretty printing.
% In \name, record types are structural and hence any value that satisfies this
% interface is of type \lstinline$IEval$ or of a subtype of \lstinline$IEval$.
% \footnote{Should be mentioned in S2.}
Having defined the interfaces, we can implement that object algebra interface
with \lstinline$evalAlg$, which is an object algebra for evaluation. 
\begin{lstlisting}{language=F2J}
let evalAlg: ExpAlg[IEval] = {
  lit = \(x: Int) -> {eval = x},
  add = \(x: IEval) (y: IEval) -> {
     eval = x.eval + y.eval
  }
};
\end{lstlisting}

In this example we implement a record, where the two operations 
\lstinline{lit} and \lstinline{add} return a record with type \lstinline{IEval}.
The type \lstinline$ExpAlg[IEval]$ is the type of object algebras
supporting evaluation. However, the one interesting point
of object algebras is that other operations can be supported as
well. 

\paragraph{Add a subtraction variant.} The point of the Expression
Problem support the addition of new features to the existing
program, without modifying existing code. 
The first feature is adding a new variant, such as subtraction. We can do so by
simply intersecting the original types and merging with the original values:
\begin{lstlisting}{language=F2J}
type SubExpAlg[E] = 
   ExpAlg[E] & {sub: E -> E -> E};
let subEvalAlg = evalAlg ,, {
   sub = \(x: IEval) (y: IEval) -> { 
      eval = x.eval - y.eval 
   }
};
\end{lstlisting}

\noindent Note that here intersection types are used to model \emph{type
  inheritance} and the merge operator models a basic form of
\emph{dynamic implementation inheritance}. 

\paragraph{Add a pretty printing operation.}
A second extension adding a new operation, such as pretty printing. 
Similar to evaluation, the interface of the pretty printing feature
is modeled as:
\begin{lstlisting}{language=F2J}
type IPrint = {print : String};
\end{lstlisting}
The implementation of pretty printing for expressions that support literals,
addition, and subtraction is:
\begin{lstlisting}{language=F2J}
let printAlg : SubExpAlg[IPrint] = {
  lit = \(x: Int) -> {print = x.toString()},
  add = \(x: IPrint) (y: IPrint) -> {
     print = x.print ++ " + " ++ y.print
  },
  sub = \(x: IPrint) (y: IPrint) -> {
     print = x.print ++ " - " ++ y.print
  }
};
\end{lstlisting}

\paragraph{Usage.}
With the definitions above, values are created using the
appropriate algebras. For example, the expression \lstinline{7 - 2} 
is encoded as follows:
\begin{lstlisting}{language=F2J}
let e1[E] (f: SubExpAlg[E]) = 
   f.sub (f.lit 7) (f.lit 2);
\end{lstlisting}

\noindent The expressions are unusual in the sense that they are
functions that take an extra argument \lstinline$f$. The extra
argument is an object algebra that uses the functions in the record
(\lstinline$lit$, \lstinline$add$ and \lstinline$sub$) as factory
methods for creating values. Moreover, the algebras themselves are
abstracted over the allowed operations such as evaluation and pretty
printing by requiring the expression functions to take an extra
argument \lstinline$E$.

\paragraph{Dynamic object algebra composition.}
To obtain an expression that supports both evaluation and pretty
printing, a mechanism to combine the evaluation and printing
algebras is needed. \name allows such composition: the \lstinline$combine$
function, which takes two object algebras to create a combined algebra. It
does so by constructing a new object algebra where each field is a
function that delegates the input to the two algebra parameters.
\begin{lstlisting}{language=F2J}
let combine[A,B](f: ExpAlg[A])(g: ExpAlg[B]) : 
  ExpAlg[A&B] = {
    lit = \(x: Int) -> f.lit x ,, g.lit x,
    add = \(x: A & B) (y: A & B) ->
          f.add x y ,, g.add x y
  }
\end{lstlisting}

\begin{lstlisting}{language=F2J}
let newAlg = 
   combine[IEval,IPrint] subEvalAlg printAlg;
let o = e1[IEval&IPrint] newAlg;
o.print ++ " = " ++ o.eval.toString()
\end{lstlisting}

Note that \lstinline$o$ is a single object that supports both
evaluation and printing. 

In contrast to the Scala solutions available in the
literature, \name is able to express object algebra
composition very directly by using the merge operator. 

\subsection{Back to Visitors}

Object Algebras are closely related to the visitor
pattern~\cite{gamma1994design}.  Indeed, object algebra interfaces are just
\emph{internal
  visitors}~\cite{oliveira09modular,oliveira2012extensibility}. 
What distinguishes object algebras
from the traditional visitor pattern is the lack of a composite
interface with an \lstinline{accept} method, which 
is both a blessing and a curse.  On
the one hand the trouble with composite interfaces with an
\lstinline{accept} method is that they make adding new variants to the
visitor pattern very hard. Although extensible versions of the visitor
pattern are possible, they usually require complex types using
advanced features of generics~\cite{togersen:2004,oliveira2012extensibility}. On the other hand, the lack of
such composite interfaces makes object algebras harder to use than
visitors. As illustrated in Section~\ref{subsec:OAs}, constructing expressions
with object algebras can only be done using a function parametrized by
an object algebra.

The remainder of the section shows that in \name there is no need to
have a dillema between extensibility using simple types and
usability: \emph{in \name it is possible to have extensible visitors with
simple types}! The key to achieve this is to have type inheritance
decoupled from subtyping, and allowing contravariant parameter type 
refinement. 

\subsubsection{The Problem with Extensible Visitors}
\begin{comment}
is another design pattern that
facilitates extensibility. The gist of this pattern is to decouple an object
hierarchy from the behaviors of each object. In other words, objects no longer
contain operations but instead ``accept'' visitors to perform operations on
them. The visitor pattern tackles at the very problem that makes programming in
traditional OO styles hard to add a new operation, since operations are defined
inside object classes that represent data structures. As a result, using the
visitor pattern allows adding new operations to existing structures without
modifying code of structures, a style enjoyed by functional
programming. 
\end{comment}
We illustrate the problem with extensible visitors using Scala.
The composite type for expressions is defined in Scala as:
\begin{lstlisting}{language=scala}
trait Exp {
  def accept(v: ExpAlg[A]): A
}
\end{lstlisting}
The trait \lstinline{Exp} has only one method 
\lstinline$accept$, which takes an internal visitor (or object
algebra) as an argument. 
Here the type \lstinline{ExpAlg[A]} is the Scala analogous of the
corresponding type defined in Section~\ref{} in \name. In terms 
of the visitor pattern, \lstinline{ExpAlg} defines the visit methods 
for all variants. 

\begin{comment}
The actual shape of the expressions
(i.e., variants) is determined by the type of the visitor, which we define in
another trait:
\begin{lstlisting}{language=scala}
trait ExpVisitor[E] {
  def lit(x: Int): E
  def add(e1: E, e2: E): E
}
\end{lstlisting}
\end{comment}

\paragraph{Adding a new variant.}
The difficulties arise when a new variant, such as subtraction is
added. To do so an extended visitor interface analogous to
\lstinline$SubExpAlg$ is needed. Moreover a corresponding composite 
interface \lstinline$SubExp$ is needed as well:
%For example, \lstinline$SubExp$ and \lstinline$SubExpAlg$
%together represent the type of expressions supporting subtraction, in addition
%to literal and addition.
\begin{lstlisting}{language=scala}
trait SubExp extends Exp {
  override def accept[E](v: SubExpAlg[E]): E
}
\end{lstlisting}
%We extend the original visitor and add the new case.
The body of \lstinline{Exp} and \lstinline{SubExp} are almost the same: they
both contain an \lstinline{accept} method that takes an object algebra 
and returns a value of the type \lstinline{E}. The only difference in
\lstinline{SubExp} the object algebra
\lstinline{v} is of type \lstinline{SubExpAlg[E]}, which is a subtype of
\lstinline{ExpAlg[E]}. 

\paragraph{Inheritance is not subtyping.}
Since \lstinline{v} appears in parameter position of
\lstinline{accept} and function parameters are naturally contravariant,
\lstinline{SubExp[E]} should be a \emph{supertype} (and not a subtype)
of \lstinline{Exp[E]}.  However, in Scala every extension induces a
subtype. In other words type inheritance and subtyping always go along
together.  To ensure type-soundness Scala (and other common OO
languages) forbid any kind of type-refinement on method parameter
types.  In other words method parameter types are invariant.
The consequence of this is that Scala is not capable of expressing
that \lstinline{SubExp[E]} is an extension and a supertype of
\lstinline{Exp}. Such kind of extension is an example where
``\emph{inheritance is not subtyping}``~\cite{cook1989inheritance}.

\begin{comment}
However,
such subtyping relation does not fit well in Scala because inheritance implies
subtyping in such languages \footnote{It is still possible to encode
  contravariant parameter types in Scala but doing so would require some
  technique.\bruno{what technique?}}. As \lstinline{SubExp[E]} extends
\lstinline{Exp[E]}, the former becomes a subtype of the latter. This suffers
from the ``Inheritance is Not Subtyping'' problem.
\end{comment}

\subsection{Extensible Visitors in \name}
Such limitation does not exist in \name. For example, we can define the similar
interfaces for \lstinline{Exp} and \lstinline{SubExp}:
\begin{lstlisting}{language=F2J}
type Exp    = {
   accept: forall A. ExpAlg[A] -> A
};
type SubExp = {
   accept: forall A. SubExpAlg[A] -> A
};
\end{lstlisting}
\name support contravariant parameter type refinement, which means that
\lstinline{SubExp} is a supertype of \lstinline{Exp}. Using these
types we first define two data constructors for simple expressions:
\begin{lstlisting}{language=F2J}
let lit (n: Int): Exp = {
  accept = /\E -> \(f: ExpAlg[E]) -> f.lit n
};
let add (e1: Exp) (e2: Exp): Exp = {
  accept = /\E -> \(f: ExpAlg[E]) -> 
      f.add (e1.accept[E] f) (e2.accept[E] f)
};
\end{lstlisting}

\noindent Both \lstinline{lit} and \lstinline{add} build values of type
\lstinline{Exp} and use object algebras of type \lstinline{ExpAlg[E]}. 
However, subtraction requires a value of type \lstinline{SubExp} to be created:
\begin{lstlisting}{language=F2J}
let sub (e1: SubExp) (e2: SubExp): SubExp ={ 
   accept = /\E -> \(f : SubExpAlg[E]) ->
        f.sub (e1.accept[E] f) (e2.accept[E] f) 
};
\end{lstlisting}

% One big benefit of using the visitor pattern is that programmers are able to
% define structures in a more natural way. 

\paragraph{Usage.} With visitors constructing expressions is quite simple:

\begin{lstlisting}{language=F2J}
e2 = sub (lit 7) (lit 2)
\end{lstlisting}

\noindent The programmer is able to pass \lstinline{lit 2}, which is of type \lstinline{Exp},
to \lstinline{sub}, which expects a \lstinline{SubExp}. The types are compatible
because because \lstinline$Exp$ is a \emph{subtype} of \lstinline$SubExp$. Code
reuse is achieved since we can use the constructors from \lstinline$Exp$ as the
constructor for \lstinline$SubExp$. In Scala, we would have to define two
literal constructors, one for \lstinline$Exp$ and another for
\lstinline$SubExp$! 

Compared to object algebras, the addition of the composite structure
allows values to be created much more intuitively, without any
drawback! All the code developed with object algebras works right
away with visitors. 

Finally note that in terms of typing, this solution does not require
any advanced use of generics. This is in sharp constrast with previous 
proposals for extensible visitors in the literature.

% \subsection{Yanlin stuff}
% \bruno{This can be dropped.}

% This subsection presents yet another lightweight solution to the Expression
% Problem, inspired by the recent work by Wang. It has been shown that
% contravariant return types allows refinement of the types of extended
% expressions.

% First, we define the type of expressions that support evaluation and implement
% two constructors:
% \begin{lstlisting}
% type Exp = { eval: Int }
% let lit (n: Int) = { eval = n }
% let add (e1: Exp) (e2: Exp)
%   = { eval = e1.eval + e2.eval }
% \end{lstlisting}

% If we would like to add a new operation, say pretty printing, it is nothing more
% than refining the original \lstinline{Exp} interface by \emph{intersecting} the
% original type with the new \lstinline{print} interface using the \lstinline{&}
% primitive and \emph{merging} the original data constructors using the \lstinline{,,}
% primitive.
% \begin{lstlisting}
% type ExpExt = Exp & { print: String }
% let litExt (n: Int) = lit n ,, { print = n.toString() }
% let addExt (e1: ExpExt) (e2: ExpExt)
%   = add e1 e2 ,,
%     { print = e1.print.concat(" + ").concat(e2.print) }
% \end{lstlisting}

% Now we can construct expressions using the constructors defined above:
% \begin{lstlisting}
% let e1: ExpExt = addExt (litExt 2) (litExt 3)
% let e2: Exp = add (lit 2) (lit 4)
% \end{lstlisting}
% \lstinline{e1} is an expression capable of both evaluation and printing, while
% \lstinline{e2} supports evaluation only.

% We can also add a new variant to our expression:
% \begin{lstlisting}
% let sub (e1: Exp) (e2: Exp) = { eval = e1.eval - e2.eval }
% let subExt (e1: ExpExt) (e2: ExpExt)
%   = sub e1 e2 ,, { print = e1.print.concat(" - ").concat(e2.print) }
% \end{lstlisting}

% Finally we are able to manipulate our expressions with the power of both
% subtraction and pretty printing.
% \begin{lstlisting}
% (subExt e1 e1).print
% \end{lstlisting}

% \subsection{Mixins}

% \bruno{Still not convinced by this section. Change to the record-based example.}
% Mixins are useful programming technique wildly adopted in dynamic programming
% languages such as JavaScript and Ruby. But obviously it is the programmers'
% responsbility to make sure that the mixin does not try to access methods or
% fields that are not present in the base class.

% In Haskell, one is also able to write programs in mixin style using records.
% However, this approach has a serious drawback: since there is no subtyping in
% Haskell, it is not possible to refine the mixin by adding more fields to the
% records. This means that the type of the family of the mixins has to be
% determined upfront, which undermines extensibility.

% \name is able to overcome both of the problems: it allows composing mixins
% that (1) extends the base behavior, (2) while ensuring type safety.

% The figure defines a mini mixin library. The apostrophe in front of types
% denotes call-by-name arguments similar to the \lstinline{=>} notation in the
% Scala language.

% \begin{lstlisting}{language=F2J}
% type Mixin S = 'S -> 'S -> S;
% let zero S (super : 'S) (this : 'S) : S = super;
% let rec mixin S (f : Mixin S) : S
%   = let m = mixin S in f (\ (_ : Unit). m f) (\ (_ : Unit). m f);
% let extends S (f : Mixin S) (g : Mixin S) : Mixin S
%   = \ (super : 'S). \ (this  : 'S). f (\ (d : Unit). g super this) this;
% \end{lstlisting}

% We define a factorial function in mixin style and make a \lstinline{noisy} mixin
% that prints ``Hello'' and delegates to its superclass. Then the two functions
% are composed using the \lstinline{mixin} and \lstinline{extends} combinators.
% The result is the \lstinline{noisyFact} function that prints ``Hello'' every
% time it is called and computes factorial.
% \begin{lstlisting}{language=F2J}
% let fact (super : 'Int -> Int) (this : 'Int -> Int) : Int -> Int
%   = \ (n : Int). if n == 0 then 1 else n * this (n - 1)
% let noisy (super : 'Int -> Int) (this : 'Int -> Int) : Int -> Int
%   = \ (n : Int). { println("Hello"); super n }
% let noisyFact = mixin (Int -> Int) (extends (Int -> Int) foolish fact)
% noisy 5
% \end{lstlisting}

% \subsection{Composing Mixins and Object Algebras}

\section{The \name calculus} \label{sec:fi}

\george{Discuss duplicate labels?}
\george{Fix highlighting or just remove it}

Following Dunfield's~\cite{dunfield2014elaborating} work on a simply-typed
lambda calculus with intersection and union types, we present the syntax,
subtyping, and typing of \name. The semantics of \name will
be defined by a type-directed translation from \name to a simple variant of
System $F$ in the next section.

\subsection{Syntax}

Figure~\ref{fig:fi-syntax} shows the syntax of \name (with the addition to
System $F$ highlighted). As a convention in this paper, we will be using
lowercase letters as meta-variables for sorts in \name, and uppercase letters
for those in the target language (starting to appear in the next section).

% To System $ F $, we add two features: intersection types and single-field
% records.
% % ~\bruno{labelled types (single field records are fine too)} 
% We include only single records because single record types as the multi-records
% can be desugared into the merge of multiple single records.

\begin{figure}[h]
  \[
\begin{array}{l}
  \begin{array}{llrl}
    \text{Types} 
    & \tau & \Coloneqq & \alpha \mid \highlight{\top} \mid \tau_1 \to \tau_2 \mid \for \alpha \tau \\
    &      & \mid      & \highlight{\tau_1 \andop \tau_2} \mid \highlight{\recty l \tau} \\
    \text{Expressions} 
    & e & \Coloneqq & x \mid \highlight{\top} \mid \lam x \tau e \mid \app e e \mid \blam \alpha e \mid \tapp e \tau \\
    &   & \mid      & \highlight {e \mergeop e} \mid \highlight {\reccon l e} \mid
                      \highlight {e.l} \mid \highlight{e \restrictop l} \\
    \text{Contexts} 
    & \gamma & \Coloneqq & \epsilon \mid \gamma, \alpha \mid \gamma, x \hast \tau \\
    \text{Labels} & l
  \end{array} 
\end{array}
\]

  \caption{Syntax of \name.}
  \label{fig:fi-syntax}
\end{figure}\bruno{I am not sure if the highlighting will be
  visible. Use gray?}\bruno{there is no \lstinline{with} anymore.
Make sure the syntax is consistent with what is presented in the type rules!}

Meta-variables $\tau$ range over types. Types include System $F$ constructs:
type variables $\alpha$; function types $\tau_1 \to \tau_2$; and type
abstraction $ \for \alpha \tau $. The top type $\top$ is the supertype of all
types. It is in fact the $0$-ary intersection. $\tau_1 \andop \tau_2$ denotes
the intersection of type $\tau_1$ and $\tau_2$; and $\recty l \tau$ the types
for single-field records. Single-field record types can be viewed as types
tagged with a label $l$, while the other type forms are untagged types. We omit
type constants such as \lstinline$Int$ and \lstinline$String$.
% \bruno{be consistent: in the source language we use \&, and in the
% formalization we use $\andop$}

Expressions include standard constructs in System
$F$: variables $ x $; abstraction of expressions over variables of a given type
$\lam x \tau e$ (concretely written \lstinline$\(x:T) -> e$); 
abstraction of expressions over types $\blam \alpha e$ (concretely written \lstinline$/\A -> e$); 
application of expressions to expressions $\app {e_1} {e_2}$; 
and application of expressions to types $\tapp e \tau$ (concretely written \lstinline$e[T]$). 
The last four constructs are new.
The canonical term that inhabits the top type is also written as $\top$.
$e_1 \mergeop e_2$ is the \emph{merge} of two expressions $e_1$ and $e_2$.
% \bruno{explain further, after all this is new.}
It can be used as either $ e_1 $ or $ e_2 $. In particular, if one regards $e_1$
and $e_2$ as objects, their merge will respond to every method that one or
both of them have. Merge of expressions correspond to intersection types
$ \tau_1 \andop \tau_2 $. The expression $ \reccon l e $ constructs a
single-field record, whereas $ e.l $ accesses the field labelled $ l $ in $ e $. 
\begin{comment}
Note that $ e $ does not
need to be a record type in this case. For example, although the merge of two
records
\[
x = \reccon {l_1} {e_1} \mergeop \reccon {l_1} {e_2} 
\]
is of an intersection type, $ x.{l_1} $ still gives $ e_1 $. On the other hand,
$ x.{l_3} $ will be rejected by the type system. 
\end{comment}
Restriction $e \restrictop l$ removes the field $l$ inside $e$. In order to
focus on the most essential features, we do not include other forms such as
fixpoints here, although they are supported in our implementation and can
be included in formalization in standard ways.

Typing contexts $ \gamma $ track bound type variables and variables (and their
type $\tau$). We use $l$ for labels, whose nature is left undefined. We use
$\subst {\tau_1} \alpha {\tau_2}$ for the capture-avoiding substitution of
$\tau_1$ for $\alpha$ inside $\tau_2$ and $\ftv \cdot$ for sets of free
variables.

\begin{comment}
\paragraph{Discussion.} A natural question the reader might ask is that why we
have excluded union types from the language. The answer is we found that
intersection types alone are enough support extensible designs. To focus on the
key features that make this language interesting, we also omit other common
constructs. For example, fixpoints can be added in standard ways.
% Dunfield has described a language that includes a ``top'' type but it does not
% appear in our language. Our work differs from Dunfield in that ...
\end{comment}

\subsection{Subtyping}

\begin{comment}
In some calculi, the subtyping relation is external to the language: those
calculi are indifferent to how the subtyping relation is defined. In \name, we
take a syntatic approach, that is, subtyping is due to the syntax of types.
However, this approach does not preclude integrating other forms of subtyping
into our system. One is ``primitive'' subtyping relations such as natural
numbers being a subtype of integers. The other is nominal subtyping relations
that are explicitly declared by the programmer.
\end{comment}

\begin{figure*}
  \small
  \framebox{$\tau \subtype \tau$}

\begin{mathpar}
\inferrule* [right=$\rulelabelsubvar$]
  { }
  {\alpha \subtype \alpha}

\inferrule* [right=$\rulelabelsubtop$]
  { }
  {\tau \subtype \top}

\inferrule* [right=$\rulelabelsubfun$]
  {\tau_3 \subtype \tau_1 \andalso \tau_2 \subtype \tau_4}
  {\tau_1 \to \tau_2 \subtype \tau_3 \to \tau_4}

\inferrule* [right=$\rulelabelsubforall$]
  {\tau_1 \subtype \subst {\alpha_1} {\alpha_2} \tau_2}
  {\for {\alpha_1} \tau_1 \subtype \for {\alpha_2} \tau_2}

\inferrule* [right=$\rulelabelsuband$]
  {\tau_1 \subtype \tau_2 \andalso \tau_1 \subtype \tau_3}
  {\tau_1 \subtype \tau_2 \andop \tau_3}

\inferrule* [right=$\rulelabelsubandleft$]
  {\tau_1 \subtype \tau_3}
  {\tau_1 \andop \tau_2 \subtype \tau_3}

\inferrule* [right=$\rulelabelsubandright$]
  {\tau_2 \subtype \tau_3}
  {\tau_1 \andop \tau_2 \subtype \tau_3}

\inferrule* [right=$\rulelabelsubrec$]
  {\tau_1 \subtype \tau_2}
  {\recty l {\tau_1} \subtype \recty l {\tau_2}}
\end{mathpar}
  \caption{Subtyping in \name.}
  \label{fig:fi-subtyping}
\end{figure*}

The syntax-directed subtyping rules of \name are shown in Figure~\ref{fig:fi-subtyping}. They
are not very surprising. The rule $\rulelabelsubfun$ says that functions are
contravariant in their parameter type and covariant in their return type. A
universal quantifier ($\forall$) is covariant in its body. A single-field record
type is also covariant, which becomes obvious if we regard it as just a labelled
type. The three rules dealing with intersection types are just what one would
expect when interpreting types as sets. Under this interpretation, for example,
the rule $\rulelabelsuband$ says that if $\tau_1$ is both the subset of $\tau_2$
and $\tau_3$, then $\tau_1$ is also the subset of the intersection of $\tau_2$
and $\tau_3$.

% Intersection types introduce natural subtyping relations among types. For
% example, $ \Int \andop \Bool $ should be a subtype of $ \Int $, since the former
% can be viewed as either $ \Int $ or $ \Bool $. To summarize, the subtyping rules
% are standard except for three points listed below:
% \begin{enumerate}
% \item $ \tau_1 \andop \tau_2 $ is a subtype of $ \tau_3 $, if \emph{either} $ \tau_1 $ or
%   $ \tau_2 $ are subtypes of $ \tau_3 $,

% \item $ \tau_1 $ is a subtype of $ \tau_2 \andop \tau_3 $, if $ \tau_1 $ is a subtype of
%   both $ \tau_2 $ and $ \tau_3 $.

% \item $ \recty {l_1} {\tau_1} $ is a subtype of $ \recty {l_2} {\tau_2} $, if
%   $ l_1 $ and $ l_2 $ are identical and $ \tau_1 $ is a subtype of $ \tau_2 $.
% \end{enumerate}
% The first point is captured by two rules $ \rulelabelsubandleft $ and
% $ \rulelabelsubandright $, whereas the second point by $ \rulelabelsuband $.
% Note that the last point means that record types are covariant in the type of
% the fields.

It is easy to see that subtyping is reflexive and transitive.

\begin{lemma}[Subtyping is reflexive] \label{sub-refl}
Given a type $ \tau $, $ \tau \subtype \tau $.
\end{lemma}

\begin{lemma}[Subtyping is transitive] \label{sub-trans}
If $ \tau_1 \subtype \tau_2 $ and $ \tau_2 \subtype \tau_3 $,
then $ \tau_1 \subtype \tau_3 $.
\end{lemma}

For the corresponding mechanized proofs in Coq, we refer to the supplementary
materials submitted with the paper. \bruno{State the reflexivity and
  transitivity Lemmas here then!}

\subsection{Typing}

\begin{figure*}
  \small
  % Typing 
\begin{mathpar}
  \framebox{$ \judgee \gamma e \tau $} \and
  \ruleevar \and \ruleetop \and
  \ruleelam  \and \ruleeapp \and
  \ruleeblam \and \ruleetapp \and
  \ruleemerge \and
  \ruleerecconstruct \and
  \ruleerecselect \and \ruleerecrestrict
\end{mathpar}


% Selection
\begin{mathpar}
  \framebox{$\judgeselect {\tau_1} l \tau_2 $} \and
  \ruleget \and \rulegetleft \and \rulegetright
\end{mathpar}


% Restriction
\begin{mathpar}
  \framebox{$\judgerestrict {\tau_1} l \tau_2 $} \and
  \rulerestrict \and \rulerestrictleft \and \rulerestrictright
\end{mathpar}


% Update
% \begin{mathpar}
%   \framebox{$\judgeupdate \tau l \tau {\tau_2} {\tau_3}$} \and
%   \ruleupdate \and \ruleupdateleft \and \ruleupdateright
% \end{mathpar}
  \caption{The type system of \name.}
  \label{fig:fi-typing}
\end{figure*}
\bruno{Please make the rule forms appear slightly above the rules.}

The syntax-directed typing rules of \name are shown in Figure~\ref{fig:fi-typing}. They consists of one
main typing judgment and two auxiliary judgments. The main typing judgment is of
the form: $ \judgee \gamma e \tau $. It says that ``in the typing context
$\gamma$, the expression $e$ is of type $\tau$''. The rules that are the same as
in System $F$ are rules for variables ($\rulelabelevar$), lambda abstractions
($\rulelabelelam$), type abstraction ($\rulelabeleblam$), and type application
($\rulelabeletapp$). The rule $\rulelabeleapp$ needs special attention as we add
a subtyping requirement in the premise: the type of the argument ($\tau_3$) is a
subtype of that of the parameter ($\tau_1$). 
%The advantage is that it then
%becomes easier to derive an algorithm for typechecking. 
For merges
$e_1 \mergeop e_2$, we typecheck $e_1$ and $e_2$ respectively, and give it the
intersection of the resulting types $\tau_1 \andop \tau_2$. The rule for
single-field record construction ($\rulelabelerecconstruct$) is standard. The rules
for record selection ($\rulelabelerecselect$) and restriction
($\rulelabelerecrestrict$) are expectedly the most complicated. They turn to the
auxiliary ``select'' and ``restrict'' rules to statically check operations and
to obtain resulting types.

\paragraph{Typing record selection.}
The ``select'' judgment deals with record selection. For example,
\[
\judgeselect {\recty {\texttt{id}} \Int \andop \recty {\texttt{name}} \String}
{\texttt{name}} {\String}
\]
That means the $\texttt{name}$ field inside
$\recty {\texttt{id}} \Int \andop \recty {\texttt{name}} \String$ is of type
$\texttt{String}$. Formally, $\judgeselect {\tau_1} l {\tau_2}$ says that a field
labelled $l$ is present inside $\tau_1$ and is of type $\tau_2$. The judgment is
made of three inference rules and is recursively defined. $\rulelabelselect$ is
the base case: if we ask for a field labelled $l$ inside $\recty l \tau$, the
field is clearly present and is of type $\tau$. $\rulelabelselectleft$ and
$\rulelabelselectright$ are two symmetric step cases. Take
$\rulelabelselectleft$ for example, it means that if $l$ is present inside
$\tau_1$, then it is also present inside $\tau_1 \andop \tau_2$; and the type of
the desired field $l$ remains $\tau$ in the conclusion.

\paragraph{Typing record restriction.}
The ``restrict'' judgment $\judgerestrict {\tau_1} l {\tau_2}$ deals with record
restriction and is very similar to the ``select'' judgment. The only difference
is that instead of giving the type of field being selected, the judment holds
the type after the restriction. For example,
\[
\judgerestrict {\recty {\code{id}} \Int \andop \recty {\code{name}} \String}
{\code{name}} {\recty {\code{id}} \Int \andop \top}
\]
Note that $\recty {\code{id}} \Int \andop \top$, interpreted set-theoretically,
is equivalent to $\recty {\code{id}} \Int$. Indeed, our system is able to judge
they are subtype of each other.

\bruno{We should probably say that {id:Int} \& T is equivalent to {id:Int}}

% The last two rules make use of the $ \rulename{fields} $ function just to make
% sure that the field being accessed ($ \rulelabelerecselect $) or updated
% ($ \rulelabelerecupd $) actually exists. The function is defined recursively, in
% Haskell pseudocode, as:
% \[ \begin{array}{rll}
%   \fields{\alpha} & = & \rel{\cdot} {\alpha} \\
%   \fields{\tau_1 \to \tau_2} & = & \rel{\cdot} {\tau_1 \to \tau_2} \\
%   \fields{\forall \alpha. \tau} & = & \rel{\cdot} {\forall \alpha. \tau} \\
%   \fields{\tau_1 \andop \tau_2} & = & \fields{\tau_1} \dplus \fields{\tau_2} \\
%   \fields{\recty l \tau} & = & \rel l t
% \end{array} \]
% where $ \cdot $ means empty list, $ \dplus $ list concatenation, and $ : $ is an
% infix operator that prepend the first argument to the second. The function
% returns an associative list whose domain is field labels and range types.

\section{Type-directed Translation to System $ F $}

In this section we define the dynamic semantics of the call-by-value \name by
means of a type-directed translation to a variant of System $F$. This
translation turns merges into usual pairs, similar to Dunfield's elaboration
approach~\cite{dunfield2014elaborating}. But in addition, our translation
removes labels of records and rewrites record operations as function
applications. In the end the translated expressions can be typed and interpreted
within System $F$.

\subsection{Informal Discussion}

This subsection presents the translation informally by explaining the major
ideas. 

\paragraph{Turning merges into pairs.}
The first idea is turning merges into pairs. For example,
\[
1 \mergeop \code{"one"}
\]
becomes \pair 1 {\code{"one"}}.
In usage, the pair will be coerced according to type information. For example,
consider the function application:
\[
\app {(\lam x \String x)} {(1 \mergeop \code{"one"})}
\]
It will be translated to
\[
\app {(\lam x \String x)} {(\app {(\lam x {\pair \Int \String} {\proj 2 x})} {\pair 1 {\code{"one"}}})}
\]
The coercion in this case is $(\lam x {\pair \Int \String} {\proj 2 x})$.

\noindent The coercion extracts the second item from the pair since the function expects a $\String$
but the translated argument is of type $\pair \Int \String$. 

\paragraph{Erasing labels.}
The second idea is erasing record labels. For example,
\begin{lstlisting}
{name = "Barbara"}
\end{lstlisting}
becomes just \lstinline{"Barbara"}.
To see how the this and the previous idea are used together, consider the following program:
\begin{lstlisting}
{distance = {inKilometers = 8, inMiles = 5}}
\end{lstlisting}
Since multi-field records are just merges, the record is desugared as
\begin{lstlisting}
{distance = {inKilometers = 8} ,, {inMiles = 5}}
\end{lstlisting}
and then translated to \lstinline{(8,5)}.

\paragraph{Record operations as functions.}
The third idea is translating record operations into normal functions. For
example, the source program
\begin{lstlisting}
{distance = {inKilometers = 8, inMiles = 5}}.distance.inMiles
\end{lstlisting}
becomes
\[
\app {(\lam x {\pair \Int \Int} {\proj 2 x})} {\pair 8 5}
\]
where $\lam x {\pair \Int \Int} {\proj 2 x}$
extracts the desired item $5$.

\subsection{Target Language}

Our target language is System $F$ extended with pair and unit types. The syntax
and typing is completely standard. The syntax of the target language is shown in
Figure~\ref{fig:f-syntax} and the typing rules in the Appendix~\george{cross
  ref}.
% \bruno{fill!}
\begin{figure}[h]
  \[
\begin{array}{llrl}
  \text{Types}       & T    & \Coloneqq & \alpha \mid () \mid T \to T \mid \for \alpha T \mid \pair T T \\ 
  \text{Expressions} & E, C & \Coloneqq & x \mid () \mid \lam x T E \mid \app E E \mid \blam \alpha E \\ 
                     &      & \mid      & \tapp E T \mid \pair E E \mid \proj k E \\
  \text{Contexts} & \Gamma & \Coloneqq & \epsilon \mid \Gamma, \alpha \mid \Gamma, x \hast T \\
\end{array}
\]

  \caption{Target language syntax.}
  \label{fig:f-syntax}
\end{figure}

% \bruno{Why is this lemma placed here?}
% \bruno{Generaly Speaking this text seems out of place.Move to 5.4, maybe?}

% The main translation judgment is $ \judgee \gamma e \tau \yields E $ which
% states that with respect to the typing context $ \gamma $, the \name expression
% $e$ is of $\tau$ and its translation is a target expression $ E $.

\subsection{Type Translation}

\begin{figure}[h]
\framebox{$\im \tau = T$}

\begin{align*}
  \im \alpha                    &= \alpha \\
  \im \top                      &= () \\
  \im {\tau_1} \to \im {\tau_2} &= \im {\tau_1} \to \im {\tau_2} \\
  \im {\for \alpha \tau}        &= \for \alpha \im \tau \\
  \im {\tau_1 \andop \tau_2}    &= \pair {\im {\tau_1}} {\im {\tau_2}} \\
  \im {\recty l \tau}           &= \im \tau
\end{align*}
\framebox{$\im \gamma = \Gamma$}

\begin{align*}
  \im \epsilon                    &= \epsilon \\
  \im {\gamma, \alpha}            &= \im \gamma, \alpha \\
  \im {\gamma, \alpha \hast \tau} &= \im \gamma, \alpha \hast \im \tau
\end{align*}
\caption{Type and context translation.}
\label{fig:type-and-context-translation}
\end{figure}

Figure~\ref{fig:type-and-context-translation} defines the type translation
function $\im \cdot$ from \name types $\tau$ to target language types $T$. The
notation $\im \cdot$ is also overloaded for context translation from \name
contexts $\gamma$ to target language contexts $\Gamma$.

% The rules given in this section are identical with those in
% Section~\ref{sec:fi}, except for the light blue part. The translation consists
% of four sets of rules, which are explained below:

\subsection{Coercive Subtyping}

\begin{figure*}
  \small
  \begin{mathpar}
\framebox{$ \tau \subtype \tau \yields C $}

\rulesubvar

\rulesubtop

\rulesubfun

\rulesubforall

\rulesuband

\rulesubandleft

\rulesubandright

\rulesubrec

\end{mathpar}
  \caption{Coercive subtyping.}
  \label{fig:elab-subtyping}
\end{figure*}

Figure~\ref{fig:elab-subtyping} shows subtyping with coercions. The judgment
\[
\tau_1 \subtype \tau_2 \yields C
\]
extends the subtyping judgment in Figure~\ref{fig:fi-subtyping} with a coercion
on the right hand side of $ \yields {} $. A coercion $ C $ is just an expression
in the target language and is ensured to have type
$ \im {\tau_1} \to \im {\tau_2} $ (Lemma~\ref{type-coerce})\bruno{ref
  now showing}. For example,
\[
\Int \andop \Bool \subtype \Bool \yields {\lam x {\im {\Int \andop \Bool}} {\proj 2 x}}
\]

\noindent generates a coercion function from $\Int \andop \Bool$ to $\Bool$.

In rules $\rulelabelsubvar$, $\rulelabelsubtop$, $\rulelabelsubforall$,
coercions are just identity functions. In $\rulelabelsubfun$, we elaborate the
subtyping of parameter and return types by $\eta$-expanding $f$ to
$\lam x {\im {\tau_3}} {\app f x}$, applying $C_1$ to the argument and $C_2$ to
the result. Rules $\rulelabelsubandleft$, $\rulelabelsubandright$, and
$\rulelabelsuband$ elaborate with intersection types. $\rulelabelsuband$ uses
both coercions to form a pair. Rules $\rulelabelsubandleft$ and
$\rulelabelsubandright$ reuse the coercion from the premises and create new ones
that cater to the changes of the argument type in the conclusions. Note that the
two rules are syntatically the same and hence a program can be elaborated
differently, depending on which rule is used. But in the implementation one
usually applies the rules sequentially with pattern matching, essentially
defining a deterministic order of lookup.
\begin{comment}
if we know $\tau_1$ is a subtype of $\tau_3$ and $C$ is a coercion from $\tau_1$
to $\tau_3$, then we can conclude that $\tau_1 \andop \tau_2$ is also a subtype
of $\tau_3$ and the new coercion is a function that takes a value $ x $ of type
$\tau_1\andop \tau_2$, project $x$ on the first item, and apply $ C $ to it.
\end{comment}

\begin{lemma}[\rulelabelsub~rules produce type-correct coercion]
  If $ \tau_1 \subtype \tau_2 \yields C $, then $ \judget \epsilon C {\im {\tau_1} \to \im {\tau_2}} $.
\end{lemma}

\begin{proof}
  By a straighforward induction on the derivation.
\end{proof}

\subsection{Main Translation}

\begin{comment}
In this subsection we now present formally the translation rules that convert
\name expressions into System $ F $ ones. This set of rules essentially extends
those in the previous section with the light-blue part for the translation.
\end{comment}

\begin{figure*}
  \small
  % Typing 
\begin{mathpar}
  \framebox{$ \judgee \gamma e \tau \yields E $} \and
  \ruleevarelab \and \ruleetopelab \and
  \ruleelamelab  \and \ruleeappelab \and
  \ruleeblamelab \and \ruleetappelab \and
  \ruleemergeelab \and
  \ruleerecconstructelab \and
  \ruleerecselectelab \and \ruleerecrestrictelab
\end{mathpar}


% Selection
\begin{mathpar}
  \framebox{$\judgeselect {\tau_1} l \tau_2 \yields C$} \and
  \rulegetelab \and \rulegetleftelab \and \rulegetrightelab
\end{mathpar}


% Restriction
\begin{mathpar}
  \framebox{$\judgerestrict {\tau_1} l \tau_2 \yields C$} \and
  \rulerestrictelab \and \rulerestrictleftelab \and \rulerestrictrightelab
\end{mathpar}


% Update
% \begin{mathpar}
%   \framebox{$\judgeupdate \tau l {\tau \yields E} {\tau_2} {\tau_3} \yields C$} \and
%   \ruleupdateelab \and \ruleupdateleftelab \and \ruleupdaterightelab
% \end{mathpar}
  \caption{Elaboration typing from \name to System $ F $.}
\end{figure*}

% \bruno{Badly structured. Don't mention Coercion here, as it was already
% explained in the previous section.}
% \bruno{Don't use itemize and items. Use paragraphs instead!}

\paragraph{Main translation judgment.} The main translation judgment
$\judgee \gamma e \tau \yields E$ extends the typing judgment with an elaborated
expression on the right hand side of $\yields {}$. The translation ensures
that $E$ has type $\im \tau$. In \name, one may pass more information to a
function than what is required; but not in System $F$. To account for this
difference, in $\rulelabeleapp$, the coercion $C$ from the subtyping relation is
applied to the argument. $\rulelabelemerge$ straighforwardly translates merges
into pairs. As record labels are erased, $\rulelabelerecconstruct$ yields the
same target expression $E$ from the premise. 

\begin{comment}
In $\rulelabelerecselect$ and $\rulelabelerecrestrict$ the coercions generated
by the ``select'' and ``restrict'' rules will be used to coerce the main \name
expression.
\end{comment}

$\rulelabelerecselect$ typechecks $e$ and use the ``select'' rule to return the
type of the field $\tau_1$ and the coercion $C$. The type of the whole expression
is $\tau_1$ and its translation of $\app C E$. $\rulelabelerecrestrict$ is
exactly the same as $\rulelabelerecselect$ except that it uses the auxiliary
``restrict'' rule. 

\paragraph{``Select'' judgment.} The ``select'' judgment additionally generates a
coercion on the right-hand side of $\yields {}$, which can be thought as a field
selector in the target language. For example, in translating the \name
expression
\[
\reccon {\code{id}} {12}.\code{id}
\]
the judgment
\[
\judgeselect {\recty {\code{id}} \Int} {\code{id}} {\Int} \yields {\lam x {\im {\recty {\code{id}} \Int}} x}
\]
gives a ``selector'' $\lam x {\im {\recty {\code{id}} \Int}} x$ that can be
applied to the translation of $\reccon {\code{id}} {12}$. The generation of
selectors is defined recursively. $\rulelabelselect$ is the base case: since the
type of the field labelled $ l $ in a $\recty l \tau$ is just $ \tau $, the
coercion is an identity function. $\rulelabelselectleft$
and $\rulelabelselectright$ builds selectors for intersection types
$\tau_1 \andop \tau_2$ according to the selector for $\tau_1$ or $\tau_2$. The
same idea appears in the twin $\rulelabelsubandleft$ and
$\rulelabelsubandright$.

\begin{lemma}[\rulelabelselect~rules produce type-correct coercion] \label{lemma:select-correct}
  If $ \judgeselect \tau l {\tau_1} \yields C $, then $ \judget \epsilon C {\im \tau \to \im {\tau_1}} $.
\end{lemma}

\begin{proof}
  By structural induction of the derivation.
\end{proof}

% Consider the source program:
% \begin{lstlisting}
%   ({ name = "Isaac", age = 10 }).name
% \end{lstlisting}

%   Multi-field records are desugared into merge of single-field records:
%   \begin{lstlisting}
%     ({ name = "Isaac"} ,, { age = 10 }).name
%   \end{lstlisting}

%   By $ \rulelabelselect $,
%   \[ \turnsget (\recty {name} {String}; {name}) : String \]

%   we have the coercion
%   \[ \lam x {\im {\recty {name} {String}}} x \]

%   which is just $ \lam x {String} x $ according to type translation.

%   By $ \rulelabelselectleft $,
%   \[ \turnsget (\recty {name} {String} \andop \recty {age} {Int}; {name}) : String \]

%   % we have the coercion
%   % \[ \abs {\rel x {\im {\recty {name} {String} \andop \recty
%   %         {age} {Int}}}} \app {(\abs {\rel x {\im {\recty {name} {String}}}} x)} {(\fst ~ x)} \]
%   % which is just $ \abs {\rel x {(String, Int)}} {\app {(\abs {\rel x {String}} x)} {(\fst ~ x)}} $ by type translation.

%   By typing rules, the translation of the program is
%   \[ ("Isaac", 10) \]. If we apply the coercion to it, we get
%   \[ "Isaac" \]

\paragraph{``Restrict'' judgment.}
The ``restrict'' judgment deals with record restriction. 
The rules are analogous to the ``select'' rules.
Compared with the coercions generated by the ``select'' rules, the coercions
generated here keep all but the restricted field in an expresison. In the base
case ($\rulelabelrestrict$), removing a field labelled $l$ from a single-field
record with the same label should result in the top value. Therefore, the
coercion is a constant function that returns unit, which is just the image of
top value in the target language. For the case
\[
\reccon {\code{name}} {\code{"Alan"}} \andop \reccon {\code{age}} {24} \restrictop \code{name}
\]
the coercion will keep the $\code{name}$ field and replace the $\code{age}$
field with a unit.

\begin{lemma}[\rulelabelrestrict~rules produce type-correct coercion] \label{lemma:restrict-correct}
  If $ \judgerestrict \tau l {\tau_1} \yields C $, then $ \judget \epsilon C {\im \tau \to \im {\tau_1}} $.
\end{lemma}

\begin{proof}
  By structural induction of the derivation.
\end{proof}

\newcommand{\crestrictone}{\lam \_ {\im {\recty \J \Int}} {()}}
\newcommand{\crestricttwo}{\lam x {\im {\recty \I \Int \andop \recty \J \Int}} {\pair {\proj 1 x} {\  \app {(\crestrictone)} {\proj 2 x}}}}

\begin{comment}
To illustrate the idea of translation in more detail, we show a step-by-step
derivation in Figure~\ref{fig:derivation} of translating the \name program
program:
\[
\small
\reccon \I 0 \mergeop \reccon \J 0 \restrictop \J
\]
into the target language. After evaluation, the target expression becomes just
$\pair 0 {()}$.

  \begin{figure*}[h]
    \small
    \begin{mathpar}
      \small

      \inferrule* [right=$\rulelabelrestrictright$]
      {\judgerestrict {\recty \J \Int} \J \top \yields {\lam \_ {\im {\recty \J \Int}} {()}}}
      {\judgerestrict {\recty \I \Int \andop \recty \J \Int} \J {\recty \I \Int \andop \top} \yields {\crestricttwo}}

      \inferrule* [right=$\rulelabelerecrestrict$]
      {      
        \inferrule* [right=$\rulelabelemerge$]
        {
          \inferrule* [right=$\rulelabelerecconstruct$]
          {\ldots}
          {\judgee \epsilon {\reccon \I 0} {\recty \I \Int} \yields 0}
          \\
          \inferrule* [right=$\rulelabelerecconstruct$]
          {\ldots}
          {\judgee \epsilon {\reccon \J 0} {\recty \J \Int} \yields 0}
        }
        {\judgee \epsilon {\reccon \I 0 \mergeop \reccon \J 0} {\recty \I \Int \andop \recty \J \Int} \yields {\pair 0 0}}
        \\
        \ldots
      }
      {\judgee \epsilon {\reccon \I 0 \mergeop \reccon \J 0 \restrictop \J} {\recty \I \Int \andop \top} \yields {\app {(\crestricttwo)} {\pair 0 0}}}
    \end{mathpar}

    \caption{An example of translating record restriction.}
    \label{fig:derivation}
  \end{figure*}
\end{comment}

\begin{theorem}[Translation preserves well-typing]
  If $ \judgee \gamma e \tau \yields E $, then $ \judget {\im \gamma} E {\im \tau} $.
\end{theorem}

\begin{proof}
(Sketch) By structural induction on the expression and the corresponding
inference rule. The full proof can be found in the appendix.
\end{proof}

Since we define the dynamic semantics of \name in terms of the composition of the type-directed translation and the dynamic semantics of System $F$, we have:

\begin{theorem}[Type safety]
  If $e$ is a well-typed \name expression, then $e$ evaluates to some System $F$
  value.
\end{theorem}
\section{Implementation}

We implemented the core functionalities of the \name as part of a JVM-based
compiler. The implementation supports record update instead of restriction as a
primitive; however the former is formalized with the same underlying idea of
elaborating records. Based on the type system of \name, we built an ML-like
source language compiler that offers interoperability with Java (such as object
creation and method calls). The source language is loosely based on the more
general System $F_{\omega}$ (compared to our target, System $F$) and supports a
number of other features, including multi-field records, mutually recursive
\code{let} bindings, type aliases, algebraic data types, pattern matching, and
first-class modules that are encoded with \code{letrec} and records.

Relevant to this paper are the three following phases in the compiler that
collectively turn source programs into System $F$:

\begin{enumerate}
\item A \emph{typechecking} phase that checks the usage of \name features and
  other source language features against an abstract syntax tree that follows
  the source syntax.

\item A \emph{desugaring} phase that translates well-typed source terms into
  \name terms. Source-level features such as multi-field records, type aliases
  are removed at this phase. The resulting program is just an \name expression
  extended with some other constructs necessary for code generation.

\item A \emph{translation} phase that turns well-typed \name terms into System
  $F$ ones.
\end{enumerate}

Phase 3 is what we have formalized in this paper.

\paragraph{Removing identity functions.} Our translation inserts identity
functions whenever subtyping or record operation occurs, which could mean
notable run-time overhead. But in practice this is not an issue. In the current
implementation, we introduced a partial evaluator with three simple rewriting
rules to eliminate the redundant identity functions as another compiler phase
after the translation. In another version of our implementation, partial
evaluation is weaved into the process of translation so that the unwanted
identity functions are not introduced during the translation.

\section{Related work} \label{sec:related-work}

% \url{http://homepages.inf.ed.ac.uk/gdp/publications/Sub_Par.pdf}

% \cite{plotkin1994subtyping}

% Also discussed intersection types!~\cite{malayeri2008integrating}.

% Pierce Ph.D thesis: F<: + /|
%        technical report: F + /|, closer to ours

% \cite{barbanera1995intersection}

\paragraph{Intersection types with polymorphism.}
Our type system combines intersection types and parametric polymorphism.  Closest
to us is Pierce's work~\cite{pierce1991programming1} on a prototype
compiler for a language with both intersection types, union types, and
parametric polymorphism. Similarly to \name in his system universal
quantifiers do not support bounded quantification. However Pierce did not try to prove any
meta-theoretical results and his calculus does not have a merge
operator.  Pierce has also studied a system where both intersection
types and bounded polymorphism are present in his Ph.D
dissertation~\cite{pierce1991programming2} and a 1997
report~\cite{pierce1997intersection}. Going in the direction of higher
kinds, Compagnoni and Pierce~\cite{compagnoni1996higher} add
intersection types to System $ F_{\omega} $ and use the new calculus,
$ F^{\omega}_{\wedge} $, to model multiple inheritance. In their
system, types include the construct of intersection of types of the
same kind $ K $. Davies and Pfenning
\cite{davies2000intersection} study the interactions between
intersection types and effects in call-by-value languages. And they
propose a ``value restriction'' for intersection types, similar to
value restriction on parametric polymorphism.
There have been attempts to provide a foundational calculus
for Scala that incorporates intersection
types~\cite{amin2014foundations,amin2012dependent}. 
Although the minimal Scala-like calculus does not natively support 
parametric polymorphism, it is possible to encode parametric
polymorphism with abstract type members. Thus it can be argued that 
this calculus also supports intersection types and parametric
polymorphism. However, the type-soundness of a minimal Scala-like 
calculus with intersection types and parametric polymorphism is not
yet proven. Recently, some form of intersection
types have been adopted in object-oriented languages such as Scala,
Ceylon, and Grace. Generally speaking,
the most significant difference to \name is that in all previous systems
there is no explicit introduction construct like our merge operator. As shown in
Section~\ref{sec:application}, this feature is pivotal in supporting modularity
and extensibility because it allows dynamic composition of values.

\begin{comment}
only allow intersections of concrete types (classes),
whereas our language allows intersections of type variables, such as
\texttt{A \& B}. Without that vehicle, we would not be able to define
the generic \texttt{merge} function (below) for all interpretations of
a given algebra, and would incur boilerplate code:

\begin{lstlisting}{language=haskell}
let merge [A, B] (f: ExpAlg A) (g: ExpAlg B) = {
  lit (x : Int) = f.lit x ,, g.lit x,
  add (x : A & B) (y : A & B) =
    f.add x y ,, g.add x y
}
\end{lstlisting}
\end{comment}


\paragraph{Other type systems with intersection types.}
Intersection types date back to as early as Coppo et
al.~\cite{coppo1981functional}. As emphasized throughout the paper our 
work is inspired by Dunfield~\cite{dunfield2014elaborating}. He describes a similar approach to ours:
compiling a system with intersection types into ordinary $ \lambda $-calculus
terms. The major difference is that his system does not include parametric
polymorphism, while ours does not include unions. Besides, our rules are
algorithmic and we formalize a record system.
% Although similar in spirit,
% our elaboration typing is simpler: we require subtyping in the case of
% applications, thus avoiding the subsumption rule. Besides, our treatment
% combines the merge rules ($ k $ ranges over $ \{1, 2\} $)
% \inferrule 
% {\Gamma \turns e_k : \tau}
% {\Gamma \turns e_1 \mergeop e_2 : \tau}
% and the standard intersection-introduction rule
% \inferrule 
% {\Gamma \turns e : \tau_1 \andalso \Gamma \turns e : \tau_2}
% {\Gamma \turns e : \tau_1 \andop \tau_2}
% into one rule:
% \inferrule [Merge]
% {\Gamma \turns e_1 : \tau_1 \andalso \Gamma \turns e_2 : \tau_2}
% {\Gamma \turns e_1 \mergeop e_2 : \tau_1 \andop \tau_2}
Reynolds invented Forsythe~\cite{reynolds1997design} in the 1980s. Our merge
operator is analogous to his $ p_1, p_2 $. As Dunfield
has noted, in Forsythe merges can be only used unambiguously. 
For instance, it is not allowed in Forsythe to merge two functions.

%Castagna, and Dunfield describe
%elaborating multi-fields records into merge of single-field records.
% Reynolds and Castagna do not consider elaboration and Dunfield do not
% formalize elaborating records.

% Both Pierce and Dunfield's system include a subsumption rule, which states that
% if an expression has been inferred of type $ \tau $, then it is also of any
% supertype of $ \tau $. Our system does not have this rule.

Refinement
intersection~\cite{dunfield2007refined,davies2005practical,freeman1991refinement}
is the more conservative approach of adopting intersection types. It increases
only the expressiveness of types but not terms. But without a term-level
construct like ``merge'', it is not possible to encode various language
features. As an alternative to syntatic subtyping described in this paper,
Frisch et al.~\cite{frisch2008semantic} study semantic subtyping.

\paragraph{Languages for extensibility.}
To improve support for extensibility various researchers have proposed
new OOP languages or programming mechanisms. It is interesting to
note that design patterns such as object algebras or modular visitors
provide a considerably different approach to extensibility when
compared to some previous proposals for language designs for
extensibility. Therefore the requirements in terms of type system
features are quite different.  One popular approach is \emph{family
  polymorphism}~\cite{Ernst01family}, which allows whole class hierarchies to be
captured as a family of classes. Such a family can be later reused to
create a derived family with potentially new class members, and
additional methods in the existing classes.  \emph{Virtual
  classes}~\cite{ernst2006virtual} are a concrete realization of this idea, where a
container class can hold nested inner \emph{virtual} classes (forming
the family of classes). In a subclass of the container class, the
inner classes can themselfves be \emph{overriden}, which is why they
are called virtual. There are many language mechanisms that provide
variants of virtual classes or similar mechanisms~\cite{McDirmid01Jiazzi,Aracic06CaesarJ,Smaragdakis98mixin,nystrom2006j}. The work by 
Nystrom on \emph{nested intersection}~\cite{nystrom2006j} uses a
form of intersection types to support the composition of
families of classes. Ostermann's \emph{delegation layers}~\cite{Ostermann02dynamically}
use delegation for doing dynamic composition in a system 
with virtual classes. This in contrast with most other approaches 
that use class-based composition, but closer to the dynamic
composition that we use in \name.

\begin{comment}
In contrast to type systems for virtual classes 
and similar mechanisms, the goal of our work is to study the type
systems and basic language mechanism to better support such design patterns. 


 some researchers have designed new type
system features such as virtual classes~\cite{ernst2006virtual}, polymorphic
variants~\cite{garrigue1998programming}, while others have shown employing
programming pattern such as object algebras~\cite{oliveira2012extensibility} by
using features within existing programming languages. Both of the two approaches
have drawbacks of some kind. The first approach often involves heavyweight
designs, while the second approach still sacrifices the readability for
extensibility.
\bruno{fill me in with more details and more references!}
\end{comment}

% Intersection types have been shown to be useful in designing languages that
% support modularity.~\cite{nystrom2006j}

\paragraph{Extensible records.}

%\george{Record field deletion is also possible.}

% http://elm-lang.org/learn/Records.elm

Encoding records using intersection types appear in
Reynolds~\cite{reynolds1997design} and Castagna et
al.~\cite{castagna1995calculus}. Although Dunfield also discusses this idea in
his paper \cite{dunfield2014elaborating}, he only provides an implementation but
not formalization. Very similar to our treatment of elaborating records is
Cardelli's work~\cite{cardelli1992extensible} on translating a calculus, named
$ F_{\subtype \rho}$, with extensible records to a simpler calculus that without
records primitives (in which case is $ F_{\subtype} $). But he does not consider
encoding multi-field records as intersections; hence his translation is more
heavyweight. Crary~\cite{crary1998simple} uses intersection types and
existential types to address the problem that arises when interpreting method
dispatch as self-application. But in his paper, intersection types are not used
to encode multi-field records.

Wand~\cite{wand1987complete} started the work on extensible records and proposes
row types~\cite{wand1989type} for records. Cardelli and
Mitchell~\cite{cardelli1990operations} defined three primitive operations on
records that are similar to ours: \emph{selection}, \emph{restriction}, and
\emph{extension}. The merge operator in \name plays the same role as extension.
Following Cardelli and Mitchell's approach,
Leijen~\cite{leijen2004first,leijen2005extensible} define record update in terms
of restriction and extension. Both Leijen's system and ours allows records that contain
duplicate labels. Arguably Leijen's system is stronger. For example, it supports
passing record labels as arguments to functions. He also shows encoding an
intersection types using first-class labels. \bruno{check carefully
  this text!}
Chlipala's
\texttt{Ur}~\cite{chlipala2010ur} explains record as type level
constructs.\bruno{What is the point of citing Chlipala's paper?}
\begin{comment}
Our system can be adapted to simulate systems that support extensible
records but not intersection of ordinary types like \texttt{Int} and
\texttt{Float} by allowing only intersection of record types.

$ \turnsrec \tau $ states that $ \tau $ is a record type, or the intersection of
record types, and so forth.

\inferrule [RecBase] {} {\turnsrec \recty l \tau}

\inferrule [RecStep]
{\turnsrec \tau_1 \andalso \turnsrec \tau_2}
{\turnsrec \tau_1 \andop \tau_2}

\inferrule [Merge']
{\Gamma \turns e_1 : \tau_1 \yields {E_1} \andalso \turnsrec \tau_1 \\
 \Gamma \turns e_2 : \tau_2 \yields {E_2} \andalso \turnsrec \tau_2}
{\Gamma \turns e_1 \mergeop e_2 : \tau_1 \andop \tau_2 \yields {\pair {E_1} {E_2}}}

Of course our approach has its limitation as duplicated labels in a record are
allowed. This has been discussed in a larger issue by
Dunfield~\cite{dunfield2014elaborating}.

R{\'e}my~\cite{remy1989type}
\end{comment}

\section{Conclusion and Further Work}

We have described a simple type system suitable for extensible designs.
The system has a term-level introduction form for intersection types, combines intersection types with
parametric polymorphism, and supports extensible records using a lightweight
mechanism. We prove that the translation is type-preserving and the language is
type-safe. 

There are various avenues for future work. On the one hand we are
interested in creating a source language where extensible designs such as object
algebras or modular visitors are supported by proper language features. On the
other hand we would like to explore extending our structural type system with
nominal subtyping to allow more familiar programming experience.


% http://en.wikibooks.org/wiki/LaTeX/Bibliography_Management

% We recommend abbrvnat bibliography style.
\bibliographystyle{abbrvnat}

% The bibliography should be embedded for final submission.
\bibliography{bib/papers}

% \begin{thebibliography}{}
% \softraggedright

% \bibitem[Smith et~al.(2009)Smith, Jones]{smith02}
% P. Q. Smith, and X. Y. Jones. ...reference text...

% Coppo, M., Dezani-Ciancaglini, M.: A new type-assignment for λ-terms. Archiv.
% Math. Logik 19, 139–156 (1978)

% \end{thebibliography}


\begin{comment}
\acks

Acknowledgments, if needed.
\end{comment}

\clearpage
\onecolumn
\appendix
\section{Type Well-formedness}

$ \ftv \cdot $ reads: ``the free type variable of''.

\begin{figure}[h]
  \framebox{$ \judgeewf \gamma \tau $}

  \begin{mathpar}
    \ruleewf
  \end{mathpar}
  \caption{Type well-formedness in \name.}
\end{figure}

\begin{figure}[h]
  \framebox{$ \judgetwf \Gamma T $}

  \begin{mathpar}
    \ruletwf
  \end{mathpar}
  \caption{Type well-formedness in the target type system.}
\end{figure}

\section{Target Type System}

\begin{figure}[h]
  \framebox{$ \judget \Gamma E T $}
  \begin{mathpar}

    \ruletvar

    \ruletunit

    \ruletlam

    \ruletapp

    \ruletblam

    \rulettapp

    \ruletpair

    \ruletprojleft

    \ruletprojright

  \end{mathpar}

  \caption{Target type system.}
\end{figure}
\section{Proofs}

\paragraph{Notation.} We sketch our proofs in two-column style: on the left are
the intermediate results and on the right are the justification (for the
previous intermediate result to reach the corresponding left-hand side). 

\subsection{Elaboration}

\bruno{fix numbering of lemmas}
\bruno{reflexitivity and transitivity missing. You can do a proof
  sketch instead of a full proof. Just say in 1 or 2 sentences what is
the main idea. You can mention that we have a full proof in Coq.}
\bruno{target type system is missing 3 cases: Tunit; Tproj1; TProj2}

\begin{lemma}[\rulelabelsub~rules produce type-correct coercion] \label{lemma:sub-correct}
  If $ \tau_1 \subtype \tau_2 \yields C $, then $ \judget \epsilon C {\im {\tau_1} \to \im {\tau_2}} $.
\end{lemma}

\begin{proof}
  By structural induction of the derivation.

  \begin{itemize}

  \item \textbf{Case}
    \begin{flalign*}
      & \rulesubvar &
    \end{flalign*}

    \begin{tabular}{rll}
      & $ \judget {\epsilon} {\lam x {\im \alpha} x} {\alpha \to \alpha} $ & By $ \rulelabeltvar $ and $ \rulelabeltlam $
    \end{tabular} \\

  \item \textbf{Case}
    \begin{flalign*}
      & \rulesubtop &
    \end{flalign*}

    \begin{tabular}{rll}
      & $ \judget {\epsilon} {\lam x {\im \tau} ()} {\im \tau \to ()} $ & By $ \rulelabeltvar $ and $ \rulelabeltlam $ \\
      & $ \judget {\epsilon} {\lam x {\im \tau} ()} {\im \tau \to \im \top} $ & By the definition of $ \im \cdot $ 
    \end{tabular} \\

  \item \textbf{Case}
    \begin{flalign*}
      & \rulesubfun &
    \end{flalign*}

    \begin{tabular}{rll}
      & $ \tau_3 \subtype \tau_1 \yields {C_1} $ & Premise \\
      & $ \judget \epsilon {C_1} {\im {\tau_3} \to \im {\tau_1}} $ & By i.h. \\
      & $ \judget \epsilon {C_2} {\im {\tau_2} \to \im {\tau_4}} $ & Similar to the above \\
      \george{TODO} &
    \end{tabular} \\

  \item \textbf{Case}
    \begin{flalign*}
      & \rulesubforall &
    \end{flalign*}

    \begin{tabular}{rll}
      & \george{TODO} &
    \end{tabular} \\

  \item \textbf{Case}
    \begin{flalign*}
      & \rulesuband &
    \end{flalign*}

    \begin{tabular}{rll}
      & \george{TODO} &
    \end{tabular} \\

  \item \textbf{Case}
    \begin{flalign*}
      & \rulesubandleft &
    \end{flalign*}

    \begin{tabular}{rll}
      & \george{TODO} &
    \end{tabular} \\

  \item \textbf{Case}
    \begin{flalign*}
      & \rulesubandright &
    \end{flalign*}

    By symmetry with the above case. \\

  \item \textbf{Case}
    \begin{flalign*}
      & \rulesubrec &
    \end{flalign*}

    \begin{tabular}{rll}
      & $ \tau_1 \subtype \tau_2 \yields C $ & Premise \\
      (a) & $ \judget \epsilon C {\im {\tau_1} \to \im {\tau_2}} $ & By i.h. \\
      & $ \judget {\epsilon, x \hast \im {\recty l {\tau_1}}} x {\im {\recty l {\tau_1}}} $ & By $ \rulelabeltvar $ \\
      & $ \judget {\epsilon, x \hast \im {\recty l {\tau_1}}} x {\im {\tau_1}} $ & By the definition of $ \im \cdot $ \\
      & $ \judget {\epsilon, x \hast \im {\recty l {\tau_1}}} {\app C x} {\im {\tau_2}} $ & By $ \rulelabeltapp $ and (a) \\
      & $ \judget {\epsilon, x \hast \im {\recty l {\tau_1}}} {\app C x} {\im {\recty l {\tau_2}}} $ & By the definition of $ \im \cdot $ \\
      & $ \judget \epsilon {\lam x {\im {\recty l {\tau_1}}} {\app C x}} {\im {\recty l {\tau_1}} \to \im {\recty l {\tau_2}}} $ & By $ \rulelabeltlam $ 
    \end{tabular} \\

  \end{itemize}

\end{proof}

\begin{lemma}[\rulelabelselect~rules produce type-correct coercion] \label{lemma:select-correct}
  If $ \judgeselect \tau l {\tau_1} \yields C $, then $ \judget \epsilon C {\im \tau \to \im {\tau_1}} $.
\end{lemma}

\begin{proof}
  By structural induction of the derivation.

  \begin{itemize}

  \item \textbf{Case}
    \begin{flalign*}
      & \rulegetelab &
    \end{flalign*}

    \begin{tabular}{rll}
      & $ \judget \epsilon {\lam x {\im {\recty l \tau}} x} {\im {\recty l \tau} \to \im {\recty l \tau}} $ & By $ \rulelabeltlam $ and $\rulelabeltvar$ \\
      & $ \judget \epsilon {\lam x {\im {\recty l \tau}} x} {\im {\recty l \tau} \to \im \tau} $ & By the definition of $ \im \cdot $
    \end{tabular} \\

  \item \textbf{Case}
    \begin{flalign*}
      & \rulegetleftelab &
    \end{flalign*}

    \begin{tabular}{rll}
      & $ \judget {\epsilon, x \hast \im {\tau_1 \andop \tau_2}} x {\im {\tau_1 \andop \tau_2}} $ & By $ \rulelabeltvar $ \\
      & $ \judget {\epsilon, x \hast \im {\tau_1 \andop \tau_2}} x {\pair {\im {\tau_1}} {\im {\tau_2}}} $ & By the definition of $\im \cdot$ \\
      & $ \judget {\epsilon, x \hast \im {\tau_1 \andop \tau_2}} {\proj 1 x} {\im {\tau_1}} $ & By $\rulelabeltprojleft$ \\
      & $ \judget \epsilon C {\im {\tau_1} \to \im \tau} $ & By i.h. \\
      & $ \judget {\epsilon, x \hast \im {\tau_1 \andop \tau_2}} C {\im {\tau_1} \to \im \tau} $ & By weakening \\
      & $ \judget {\epsilon, x \hast \im {\tau_1 \andop \tau_2}} {\app C {(\proj 1 x)}} {\im \tau} $ & By $\rulelabeltapp$ \\
      & $ \judget \epsilon {\lam x {\im {\tau_1 \andop \tau_2}} {\app C {(\proj 1 x)}}} {\im {\tau_1 \andop \tau_2} \to \im \tau} $ & By $ \rulelabeltlam $
    \end{tabular} \\

  \item \textbf{Case}
    \begin{flalign*}
      & \rulegetrightelab &
    \end{flalign*}

    By symmetry with the above case. \\

\end{itemize}
\end{proof}


\begin{lemma}[\rulelabelrestrict~rules produce type-correct coercion] \label{lemma:restrict-correct}
  If $ \judgerestrict \tau l {\tau_1} \yields C $, then $ \judget \epsilon C {\im \tau \to \im {\tau_1}} $.
\end{lemma}

\begin{proof}
  By structural induction of the derivation.

  \begin{itemize}

  \item \textbf{Case}
    \begin{flalign*}
      & \rulerestrictelab &
    \end{flalign*}

    \begin{tabular}{rll}
      & $ \judget \epsilon {\lam x {\im {\recty l \tau}} {()}} {\im {\recty l \tau} \to ()} $ & By $\rulelabeltunit$ and $\rulelabeltlam$ \\
      & $ \judget \epsilon {\lam x {\im {\recty l \tau}} {()}} {\im {\recty l \tau} \to \im \top} $ & By the definition of $\im \cdot$  
    \end{tabular} \\

  \item \textbf{Case}
    \begin{flalign*}
      & \rulerestrictleftelab &
    \end{flalign*}

    \begin{tabular}{rll}
      & $\judgerestrict {\tau_1} l \tau \yields C$ & Premise \\
      & $\judget \epsilon C {\im {\tau_1} \to \im \tau} $ & By i.h. \\
      & $\judget {\epsilon, x \hast \im {\tau_1 \andop \tau_2}} x {\im {\tau_1 \andop \tau_2}} $ & By $\rulelabeltvar$ \\
      & $\judget {\epsilon, x \hast \im {\tau_1 \andop \tau_2}} x {\pair {\im {\tau_1}} {\im {\tau_2}}} $ & By the definition of $\im \cdot$ \\
      & $\judget {\epsilon, x \hast \im {\tau_1 \andop \tau_2}} {\proj 1 x} {\im {\tau_1}} $ & By $\rulelabeltprojleft$ \\
      & $\judget {\epsilon, x \hast \im {\tau_1 \andop \tau_2}} {\proj 2 x} {\im {\tau_2}} $ & By $\rulelabeltprojright$ \\
      & $\judget {\epsilon, x \hast \im {\tau_1 \andop \tau_2}} {\app C {(\proj 1 x)}} {\im {\tau}} $ & By $\rulelabeltapp$ \\
      & $\judget {\epsilon, x \hast \im {\tau_1 \andop \tau_2}} {\pair {\app C {(\proj 1 x)}} {\proj 2 x}} {\pair {\im {\tau}} {\im {\tau_2}}} $ & By $\rulelabeltpair$ \\
      & $\judget {\epsilon, x \hast \im {\tau_1 \andop \tau_2}} {\pair {\app C {(\proj 1 x)}} {\proj 2 x}} {\im {\tau \andop \tau_2}} $ & By the definition of $\im \cdot$ \\
      & $\judget \epsilon {\lam x {\im {\tau_1
          \andop \tau_2}} {\pair {\app C {(\proj 1 x)}} {\proj 2 x}}} {\im {\tau_1 \andop \tau_2} \to \im {\tau \andop \tau_2}} $ & By $\rulelabeltlam$ 
    \end{tabular} \\

  \item \textbf{Case}
    \begin{flalign*}
      & \rulerestrictrightelab &
    \end{flalign*}

    By symmetry with the above case. \\

\end{itemize}
\end{proof}


\begin{lemma}[\rulelabelupdate~rules produce type-correct coercion] \label{lemma:update-correct}
  If $ \judgeupdate \tau l {\tau_1 \yields E} {\tau_2} {\tau_3} \yields C $ and $
  \judget \Gamma E {\im {\tau_1}} $ for some $ \Gamma $, then
  $ \judget \Gamma C {\im \tau \to \im {\tau_2}} $.
\end{lemma}

\begin{proof}
  By structural induction of the derivation.

  \begin{itemize}

  \item \textbf{Case}
    \begin{flalign*}
      & \ruleupdateelab &
    \end{flalign*}

    \begin{tabular}{rll}
      & $ \judget \Gamma {\lam \_ {\im {\recty l \tau}} E} {\im {\recty l \tau}} \to \im {\tau_1} $ & By $ \rulelabeltlam $, $ \rulelabeltvar $, and the hypothesis
    \end{tabular} \\

  \item \textbf{Case}
    \begin{flalign*}
      & \ruleupdateleftelab &
    \end{flalign*}

    \begin{tabular}{rll}
      & $ \judget {\Gamma, x \hast \im {\tau_1 \andop \tau_2}} x {\im {\tau_1 \andop \tau_2}} $ & By $\rulelabeltvar$ \\
      & $ \judget {\Gamma, x \hast \im {\tau_1 \andop \tau_2}} x {\pair {\im {\tau_1}} {\im {\tau_2}}} $ & By the definition of $\im \cdot$ \\
      & $ \judget {\Gamma, x \hast \im {\tau_1 \andop \tau_2}} {\proj 1 x} {\im {\tau_1}} $ & By $\rulelabeltprojleft$ \\
      & $ \judget \Gamma C {\im {\tau_1} \to \im {\tau_3}} $ & By i.h. \\ 
      & $ \judget {\Gamma, x \hast \im {\tau_1 \andop \tau_2}} C {{\im {\tau_1}} \to \im {\tau_3}} $ & By weakening \\ 
      & $ \judget {\Gamma, x \hast \im {\tau_1 \andop \tau_2}} {\app C {(\proj 1 x)}} {\im {\tau_3}} $ & By $\rulelabeltapp$ \\
      & $ \judget \Gamma {\lam x {\im {\tau_1 \andop \tau_2}} {\app C {(\proj 1 x)}}} {\im {\tau_1 \andop \tau_2} \to \im {\tau_3}} $ & By $\rulelabeltlam$
    \end{tabular} \\

  \item \textbf{Case}
    \begin{flalign*}
      & \ruleupdaterightelab &
    \end{flalign*}

    By symmetry with the above case. \\

  \end{itemize}
\end{proof}

\begin{lemma}[Preservation of well-formedness under type translation] \label{lemma:preserve-wf}
  If $ \judgeewf \gamma \tau $, then $ \judgetwf {\im \gamma} {\im \tau} $.
\end{lemma}

\begin{proof}
  By structural induction of the derivation. The only case to consider is $ \rulelabelewf $.

  \begin{itemize}

  \item \textbf{Case}

    \begin{flalign*}
      & \ruleewf &
    \end{flalign*}

    \begin{tabular}{rll}
      & $ \ftv \tau \in \gamma $ & Premise \\
      & $ \ftv {\im \tau} \in \im \gamma $ & By the definition of $ \im \cdot $ \\
      & $ \judgetwf {\im \gamma} {\im \tau} $ & By $ \rulelabeltwf $
    \end{tabular} \\

  \end{itemize}
\end{proof}

\begin{theorem}[Translation preserves well-typing]
  If $ \judgee \gamma e \tau \yields E $, then $ \judget {\im \gamma} E {\im \tau} $.
\end{theorem}

\begin{proof}
  By structural induction of the derivation.

  \begin{itemize}

  \item \textbf{Case}
    \begin{flalign*}
      & \ruleevarelab &
    \end{flalign*}

    \begin{tabular}{rll}
     & $ (x,\tau) \in \gamma $ & Premise \\
     & $ (x,\im \tau) \in \im \gamma $ & By the definition of $ \im \cdot $ \\
     & $ \judget {\im \gamma} x {\im \tau} $ & By $ \rulelabeltvar $
    \end{tabular} \\

  \item \textbf{Case}
    \begin{flalign*}
      & \ruleetopelab &
    \end{flalign*}

    \begin{tabular}{rll}
      & $\judget {\im \gamma} {()} {()} $ & By $\rulelabeltunit$ \\
      & $\judget {\im \gamma} {()} {\im \top}$ & By the definition of $ \im \cdot$ 
    \end{tabular} \\

  \item \textbf{Case}
    \begin{flalign*}
      & \ruleelamelab &
    \end{flalign*}

    \begin{tabular}{rll}
      & $ \judgee {\gamma, x \hast \tau} e {\tau_1} \yields E $ & Premise \\
      & $ \judget {\im {\gamma, x \hast \tau}} E {\im {\tau_1}} $ & By i.h. \\
      & $ \judget {\im \gamma, x \hast \im \tau} E {\im {\tau_1}} $ & By the definition of $ \im \cdot $ \\
      & $ \judget {\im \gamma} {\lam x {\im \tau} E} {\im \tau \to \im {\tau_1}} $ & By $ \rulelabeltlam $ \\
      & $ \judget {\im \gamma} {\lam x {\im \tau} E} {\im {\tau \to \tau_1}} $ & By the definition of $ \im \cdot $ 
    \end{tabular} \\

  \item \textbf{Case}
    \begin{flalign*}
      & \ruleeappelab &
    \end{flalign*}

    \begin{tabular}{rll}
     & $ \judgee \gamma {e_1} {\tau_1 \to \tau_2} \yields {E_1} $  & Premise \\
     & $ \judget {\im \gamma} {E_1} {\im {\tau_1 \to \tau_2}} $ & By i.h. \\
     & $ \judgee \gamma {e_2} {\tau_3} \yields {E_2} $ & Premise \\
     & $ \judget {\im \gamma} {E_2} {\im {\tau_3}} $ & By i.h. \\
     & $ \tau_3 \subtype \tau_1 \yields C $ & Premise \\
     & $ \judget \epsilon C {\im {\tau_3} \to \im {\tau_1}} $ & By Lemma~\ref{lemma:sub-correct} \\
     & $ \judget {\im \gamma} {\app {E_1} {(\app C E_2)}} {\im {\tau_2}} $ & By $ \rulelabeltapp $ and the definition of $ \im \cdot $
    \end{tabular} \\

  \item \textbf{Case}
    \begin{flalign*}
      & \ruleeblamelab &
    \end{flalign*}

    \begin{tabular}{rll}
      & $ \judgee {\gamma, \alpha} e \tau \yields E $ & Premise \\
      & $ \judget {\im {\gamma, \alpha}} E {\im \tau} $ & By i.h. \\
      & $ \judget {\im \gamma, \alpha} E {\im \tau} $ & By the definition of $ \im \cdot $ \\
      & $ \judget {\im \gamma} {\blam \alpha E} {\for \alpha {\im \tau}} $ & By $ \rulelabeltblam $ \\
      & $ \judget {\im \gamma} {\blam \alpha E} {\im {\for \alpha \tau}} $ & By the definition of $ \im \cdot $
    \end{tabular} \\

  \item \textbf{Case}
    \begin{flalign*}
      & \ruleetappelab &
    \end{flalign*}

    \begin{tabular}{rll}
     & $ \judgee \gamma e {\for \alpha \tau_1} \yields E $ & Premise \\
     & $ \judget {\im \gamma} E {\im {\for \alpha \tau_1}} $ & By i.h. \\
     & $ \judget {\im \gamma} E {\for \alpha \im {\tau_1}} $ & By the definition of $ \im \cdot $ \\
     & $ \judgeewf \gamma \tau $ & Premise \\
     & $ \judgetwf {\im \gamma} {\im \tau} $ & By Lemma~\ref{lemma:preserve-wf} \\
     & $ \judget \gamma {\tapp E {\im \tau}} {\subst {\im \tau} \alpha {\im {\tau_1}}} $ & By $ \rulelabelttapp $ \\
     & $ \judget \gamma {\tapp E {\im \tau}} {\im {\subst \tau \alpha {\tau_1}}} $ & By substitution lemma
    \end{tabular} \\

  \item \textbf{Case}
    \begin{flalign*}
      & \ruleemergeelab &
    \end{flalign*}

    \begin{tabular}{rll}
      & $ \judgee \gamma {e_1} {\tau_1} \yields {E_1} $ & Premise \\
      & $ \judget {\im \gamma} {E_1} {\im {\tau_1}} $ & By i.h. \\
      & $ \judget {\im \gamma} {E_2} {\im {\tau_2}} $ & Similar to the above \\
      & $ \judget {\im \gamma} {\pair {E_1} {E_2}} {\pair {\im {\tau_1}} {\im {\tau_2}}} $ & By $ \rulelabeltpair $ \\
      & $ \judget {\im \gamma} {\pair {E_1} {E_2}} {\im {\tau_1 \andop \tau_2}} $ & By the definition of $ \im \cdot $ 
    \end{tabular} \\

  \item \textbf{Case}
    \begin{flalign*}
      & \ruleerecconstructelab &
    \end{flalign*}

    \begin{tabular}{rll}
      & $ \judgee \gamma e \tau \yields E $ & Premise \\
      & $ \judget {\im \gamma} E {\im \tau} $ & By i.h. \\
      & $ \judget {\im \gamma} E {\im {\recty l \tau}} $ & By the definition of $ \im \cdot $ 
    \end{tabular} \\

  \item \textbf{Case}
    \begin{flalign*}
      & \ruleerecselectelab &
    \end{flalign*}

    \begin{tabular}{rll}
     & $ \judgeselect \tau l {\tau_1} \yields C $ & Premise \\
     & $ \judget \epsilon C {\im \tau \to \im {\tau_1}} $ & By Lemma~\ref{lemma:select-correct} \\
     & $ \judget {\im \gamma} C {\im \tau \to \im {\tau_1}} $ & By weakening \\
     & $ \judgee \gamma e \tau \yields E $ & Premise \\
     & $ \judget {\im \gamma} E {\im \tau} $ & By i.h. \\
     & $ \judget {\im \gamma} {\app C E} {\im {\tau_1}} $ & By $ \rulelabeltapp $
    \end{tabular} \\

  \item \textbf{Case}
    \begin{flalign*}
      & \ruleerecrestrictelab &
    \end{flalign*}

    \begin{tabular}{rll}
     & $ \judgerestrict \tau l {\tau_1} \yields C $ & Premise \\
     & $ \judget \epsilon C {\im \tau \to \im {\tau_1}} $ & By Lemma~\ref{lemma:restrict-correct} \\
     & $ \judget {\im \gamma} C {\im \tau \to \im {\tau_1}} $ & By weakening \\
     & $ \judgee \gamma e \tau \yields E $ & Premise \\
     & $ \judget {\im \gamma} E {\im \tau} $ & By i.h. \\
     & $ \judget {\im \gamma} {\app C E} {\im {\tau_1}} $ & By $ \rulelabeltapp $
    \end{tabular} \\

  % \item \textbf{Case}
  %   \begin{flalign*}
  %     & \ruleerecupdateelab &
  %   \end{flalign*}

  %   \begin{tabular}{rll}
  %     & $ \judgee \gamma {e_1} {\tau_1} \yields {E_1} $ & Premise \\
  %     & $ \judget {\im \gamma} {E_1} {\im {\tau_1}} $ & By i.h. \\
  %     & $ \judgeupdate \tau l {\tau_1 \yields {E_1}} {\tau_2} {\tau_3} \yields C $ & Premise \\
  %     & $ \judget {\im \gamma} C {\im \tau \to \im {\tau_2}} $ & By Lemma~\ref{lemma:update-correct} \\
  %     & $ \judgee \gamma e \tau \yields E $ & Premise \\
  %     & $ \judget {\im \gamma} E {\im \tau} $ & By i.h. \\
  %     & $ \judget {\im \gamma} {\app C E} {\im {\tau_2}} $ & By $ \rulelabeltapp $
  %   \end{tabular} \\

  \end{itemize}
\end{proof}

\end{document}
