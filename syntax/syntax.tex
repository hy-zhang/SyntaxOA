\documentclass{article}
\usepackage{amsmath,amsfonts,amsthm,amssymb,algorithm,algpseudocode}
\usepackage{color,chngpage}
\usepackage{extramarks,epsfig}
\usepackage{fancyhdr,float}
\usepackage{graphicx}
\usepackage{indentfirst}
\usepackage{lastpage,listings}
\usepackage{soul,setspace}
\usepackage{wrapfig}
\usepackage[T1]{fontenc}
\usepackage[scaled=0.85]{beramono}

% Homework Specific Information. Change it to your own

% In case you need to adjust margins:
\topmargin=-0.45in      %
\evensidemargin=0in     %
\oddsidemargin=0in      %
\textwidth=6.5in        %
\textheight=9.0in       %
\headsep=0.25in         %

\lstset{ %
language=Java,                % choose the language of the code
columns=flexible,
lineskip=-1pt,
basicstyle=\ttfamily\small,       % the size of the fonts that are used for the code
numbers=none,                   % where to put the line-numbers
numberstyle=\ttfamily\tiny,      % the size of the fonts that are used for the line-numbers
stepnumber=1,                   % the step between two line-numbers. If it's 1 each line will be numbered
numbersep=5pt,                  % how far the line-numbers are from the code
backgroundcolor=\color{white},  % choose the background color. You must add \usepackage{color}
showspaces=false,               % show spaces adding particular underscores
showstringspaces=false,         % underline spaces within strings
showtabs=false,                 % show tabs within strings adding particular underscores
frame=single,                   % adds a frame around the code
morekeywords={sig,where,type,algebra,data,from,let},
tabsize=2,                  % sets default tabsize to 2 spaces
captionpos=none,                   % sets the caption-position to bottom
breaklines=true,                % sets automatic line breaking
breakatwhitespace=false,        % sets if automatic breaks should only happen at whitespace
title=\lstname,                 % show the filename of files included with \lstinputlisting; also try caption instead of title
escapeinside={(*}{*)},          % if you want to add a comment within your code
keywordstyle=\ttfamily\bfseries,
% commentstyle=\color{Gray},
% stringstyle=\color{Green}
}

% Setup the header and footer
\pagestyle{fancy}
\rfoot{Page\ \thepage\ of\ \protect\pageref{LastPage}}
\renewcommand\headrulewidth{0.4pt}                                      %
\renewcommand\footrulewidth{0.4pt}                                      %

%%%%%%%%%%%%%%%%%%%%%%%%%%%%%%%%%%%%%%%%%%%%%%%%%%%%%%%%%%%%%
% Some tools

\newcommand{\Answer}{\ \\\textbf{Answer:} }
\newcommand{\todo}{\textbf{Proposition. } }
\newcommand{\lemma}{\textbf{Lemma. } }
\newcommand{\tab}{\ \ }
\newcommand{\red}[1]{\textcolor{red}{#1}}
\newcommand{\nb}{\textit{NB. }}

%%%%%%%%%%%%%%%%%%%%%%%%%%%%%%%%%%%%%%%%%%%%%%%%%%%%%%%%%%%%%


%%%%%%%%%%%%%%%%%%%%%%%%%%%%%%%%%%%%%%%%%%%%%%%%%%%%%%%%%%%%%
% Make title
\title{\huge\textmd{\bf Syntax: Object Algebras}}
\date{}
\author{\today}
%%%%%%%%%%%%%%%%%%%%%%%%%%%%%%%%%%%%%%%%%%%%%%%%%%%%%%%%%%%%%

\begin{document}
\begin{spacing}{1.1}
\maketitle \thispagestyle{empty}

%%%%%%%%%%%%%%%%%%%%%%%%%%%%%%%%%%%%%%%%%%%%%%%%%%%%%%%%%%%%%
% Begin edit from here

\section{Object Algebra Interface}

\subsection{Inheritance $\times$}

\subsubsection{Template}

\begin{lstlisting}[numbers=none]
BEFORE:     sig (*$N_{AI}[\overline{T_{AI}}]$*) where (*$\overline{N_{CS}:T_{CS}}$*);
\end{lstlisting}
\begin{lstlisting}[numbers=none]
AFTER:      type (*$N_{AI}[\overline{T_{AI}}]$*) = {(*$\overline{N_{CS}:T_{CS}}$*)};
\end{lstlisting}

\subsubsection{Example: \lstinline{ExpAlg[E]}}

\begin{lstlisting}[numbers=none]
BEFORE:     sig ExpAlg[E] where
                lit : Int -> E,
                add : E -> E -> E;
\end{lstlisting}
\begin{lstlisting}[numbers=none]
AFTER:      type ExpAlg[E] = {
                lit : Int -> E,
                add : E -> E -> E
            };
\end{lstlisting}

\subsection{Inheritance $\surd$}

\subsubsection{Template}

\begin{lstlisting}[numbers=none]
BEFORE:     sig (*$N_{AI}[\overline{T_{AI}}]$*) extends (*$\overline{N_{AI_2}[\overline{T_{AI_2}}]}$*) where (*$\overline{N_{CS}:T_{CS}}$*);
\end{lstlisting}
\begin{lstlisting}[numbers=none]
AFTER:      type (*$N_{AI}[\overline{T_{AI}}]$*) = (*$\red{\&}(\overline{N_{AI_2}[\overline{T_{AI_2}}]})$*) & (*$\{\overline{N_{CS}:T_{CS}}\}$*);
\end{lstlisting}

\nb How to avoid multiple inheritance from the same interface? Throw an exception if two records with same labels are combined with an ``\&''?

\subsubsection{Example: \lstinline{StatAlg[E, S]}}

\begin{lstlisting}[numbers=none]
BEFORE:     sig StatAlg[E, S] extends ExpAlg[E] where
                seq : S -> S -> S,
                asn : String -> E -> S;
\end{lstlisting}
\begin{lstlisting}[numbers=none]
AFTER:      type StatAlg[E, S] = (ExpAlg[E]) & {
                seq : S -> S -> S,
                asn : String -> E -> S
            };
\end{lstlisting}

\section{Object Algebra}

\subsection{Inheritance $\times$}

\subsubsection{Template}

\begin{lstlisting}[numbers=none]
BEFORE:     algebra (*$N_A$*) implements (*$\overline{N_{AI}[\overline{T_A}]}$*) where (*$\overline{t@(N_{CS}\ \overline{x})=E}$*);
\end{lstlisting}
\begin{lstlisting}[numbers=none]
AFTER:      let (*$N_A$*) = (*$\red{[\overline{T_A}/\overline{T_{AI}}]}$*)(*$\{\overline{N_{CS}=\lambda(\overline{x}:\overline{T_{CS}}).\ \{t=E\}}\}$*);
\end{lstlisting}

\subsubsection{Example: \lstinline{EvalExpAlg}}

\begin{lstlisting}[numbers=none]
BEFORE:     type IEval = { eval : Int };
            algebra EvalExpAlg implements ExpAlg[IEval] where
                eval@(lit x)(*$\ \ \ $*)= x,
                eval@(add x y) = x.eval + y.eval;
\end{lstlisting}
\begin{lstlisting}[numbers=none]
AFTER:      type IEval = { eval : Int };
            let EvalExpAlg = {
                lit = \(x : Int) -> { eval = x },
                add = \(x : IEval) -> \(y : IEval) -> { eval = x.eval + y.eval }
            };
\end{lstlisting}

\subsection{Inheritance $\surd$}

\subsubsection{Template}

\begin{lstlisting}[numbers=none]
BEFORE:     algebra (*$N_A$*) extends (*$\overline{N_{A_2}}$*) implements (*$\overline{N_{AI}[\overline{T_A}]}$*) where (*$\overline{t@(N_{CS}\ \overline{x})=E}$*);
\end{lstlisting}
\begin{lstlisting}[numbers=none]
AFTER:      let (*$N_A$*) =  (*$\left((\red{,,})\left(\overline{N_{A_2}}\right)\right)$*) ,, (*$\red{[\overline{T_A}/\overline{T_{AI}}]}$*)(*$\{\overline{N_{CS}=\lambda(\overline{x}:\overline{T_{CS}}).\ \{t=E\}}\}$*);
\end{lstlisting}

\nb The same for multiple inheritance.

\subsubsection{Example: \lstinline{PrintStatAlg}}

\begin{lstlisting}[numbers=none]
BEFORE:     type IPrint = { print : String };
            algebra PrintExpAlg implements ExpAlg[IPrint] where
                print@(lit x)(*$\ \ \ $*)= "\{x}",
                print@(add x y) = "\{x.print} + \{y.print}";
            algebra PrintStatAlg extends PrintExpAlg implements StatAlg[IPrint, IPrint] where
                print@(seq x y) = "\{x.print} || \{y.print}",
                print@(asn x y) = "\{x} = \{y.print}";
\end{lstlisting}
\begin{lstlisting}[numbers=none]
AFTER:      type IPrint = { print : String };
            let PrintExpAlg = {
                lit = \(x : Int) -> { print = "\{x}" },
                add = \(x : IPrint) -> \(y : IPrint) -> { print = "\{x.print} + \{y.print}" }
            };
            let PrintStatAlg = PrintExpAlg ,, {
                seq = \(x : IPrint) -> \(y : IPrint) -> { print = "\{x.print} || \{y.print}" },
                asn = \(x : String) -> \(y : IPrint) -> { print = "\{x} = \{y.print}" }
            };
\end{lstlisting}

\section{Datatype}

\subsection{Template}

\begin{lstlisting}[numbers=none]
BEFORE:     data (*$N_D[\overline{T_D}]$*) from (*$N_{AI}[\overline{T_{AI}}].S$*);
\end{lstlisting}
\begin{lstlisting}[numbers=none]
AFTER:      type (*$N_D[\overline{T_D}]$*) = { accept : forall ((*$\overline{T_{AI}}\backslash\overline{T_D}$*)). (*$N_{AI}[\overline{T_{AI}}]$*) -> (*$S$*) };
\end{lstlisting}

\nb Usually $\overline{T_{AI}}\backslash\overline{T_D}=S$.

\subsection{Example: \lstinline{List[A]}}

\begin{lstlisting}[numbers=none]
BEFORE:     sig ListAlg[A, L] where
                nil(*$\ \ $*): L,
                cons : A -> L -> L;
            data List[A] from ListAlg[A, L].L;
\end{lstlisting}
\begin{lstlisting}[numbers=none]
AFTER:      type ListAlg[A, L] = {
                nil(*$\ \ $*): L,
                cons : A -> L -> L
            };
            type List[A] = { accept : forall L. ListAlg[A, L] -> L };
\end{lstlisting}

\section{Creating a Structure}

\subsection{Template}

\subsection{Example: }

\section{Instantiation}

\subsection{Template}

\subsection{Example: }

% End edit to here
%%%%%%%%%%%%%%%%%%%%%%%%%%%%%%%%%%%%%%%%%%%%%%%%%%%%%%%%%%%%%

\end{spacing}
\end{document}

%%%%%%%%%%%%%%%%%%%%%%%%%%%%%%%%%%%%%%%%%%%%%%%%%%%%%%%%%%%%%
