\documentclass[a4paper]{article}

\usepackage[top=.5in, bottom=.5in, left=.5in, right=.5in]{geometry}

% Remote packages

% For pdflatex, replaced by fontspec:
% \usepackage{tgpagella}
\usepackage[T1]{fontenc}
\usepackage[utf8]{inputenc}

% For xelatex or lualatex:
% \usepackage{fontspec}
% \setmainfont{Times New Roman}


\usepackage{amsmath}
\usepackage{amsthm}
\usepackage{amssymb}
\usepackage{mathtools} % For \Coloneqq
\usepackage{bm}        % Bold symbols in maths mode
\usepackage{cmll}
\usepackage{fixltx2e}
\usepackage{stmaryrd}
\usepackage[dvipsnames]{xcolor}
\usepackage{listings} % For code listings
% \usepackage{minted}
% \usemintedstyle{murphy}
\usepackage{fancyvrb}
\usepackage{url}
\usepackage{xspace}
\usepackage{comment}
\usepackage{mdwlist}

% Typography
\usepackage[euler-digits,euler-hat-accent]{eulervm}

% Copied from the FCore paper:
\usepackage[colorlinks=true,allcolors=black,breaklinks,draft=false]{hyperref}   % hyperlinks, including DOIs and URLs in bibliography
% known bug: http://tex.stackexchange.com/questions/1522/pdfendlink-ended-up-in-different-nesting-level-than-pdfstartlink

% Figures with borders
% http://en.wikibooks.org/wiki/LaTeX/Floats,_Figures_and_Captions
% \usepackage{float}
% \floatstyle{boxed}
% \restylefloat{figure}

% Local packages

% \usepackage{styles/bcprules}    % by Benjamin C. Pierce
\usepackage{styles/mathpartir}  % by Didier Rémy


% ! Always load mathastext last
% http://mirrors.ibiblio.org/CTAN/macros/latex/contrib/mathastext/mathastext.pdf
% \renewcommand\familydefault\ttdefault
% \usepackage{mathastext}
% \renewcommand\familydefault\rmdefault

% http://tex.stackexchange.com/questions/114151/how-do-i-reference-in-appendix-a-theorem-given-in-the-body
\usepackage{thmtools, thm-restate}
\newtheorem{theorem}{Theorem}
\newtheorem{lemma}{Lemma}


\newcommand{\authornote}[3]{{\color{#2} {\sc #1}: #3}}
\newcommand\bruno[1]{\authornote{bruno}{red}{#1}}
\newcommand\george[1]{\authornote{george}{blue}{#1}}
\newcommand\haoyuan[1]{\authornote{bruno}{green}{#1}}

% Define macros immediately before the \begin{document} command
\newcommand{\turns}{\vdash}

\newcommand{\im}[1]{\lvert #1 \rvert}

\newcommand{\hast}{\!:\!}
\newcommand{\subst}[2]  {\lbrack #1 / #2 \rbrack}

% Relations
\newcommand{\subtype}   {<:}

\definecolor{facebook}{HTML}{3B5998}
\newcommand{\yields}[1]{\textcolor{facebook}{\; \hookrightarrow {#1}}}

% Helpers
\newcommand{\ftv}[1]{\textit{ftv}({#1})}

% Spacing
\newcommand{\binderspacing}{\,}
\newcommand{\appspacing}{\;}

% Types
\newcommand{\for}[2]{\forall #1. \binderspacing #2}
\newcommand{\recty}[2]{\{ #1 \hast #2 \}}
% \newcommand{\top}{\{\}}
\newcommand{\andop}{\with}
\newcommand{\pair}[2]{(#1, #2)}

% Expressions
\newcommand{\lam}[3]{\lambda (#1 \hast #2).\binderspacing #3}
\newcommand{\blam}[2]{\Lambda #1.\binderspacing #2}
\newcommand{\app}[2]{#1 \appspacing #2}
\newcommand{\tapp}[2]{#1 \appspacing #2}
\newcommand{\mergeop}{,,}
\newcommand{\reccon}[2]{\{ #1 = #2 \}}
\newcommand{\recupdate}[3]{#1 \; \mathbf{with} \; \{#2 = #3\}}
\newcommand{\proj}[2]{{\code{proj}}_{#1} #2}
\newcommand{\letexpr}[3]{\kwlet \; x = e \; \kwin \; e}

% Keywords
\newcommand{\keyword}[1]{\texttt{#1}}

\newcommand{\kwlet}{\keyword{let}}
\newcommand{\kwin}{\keyword{in}}
\newcommand{\kwwhere}{\keyword{where}}


\newcommand{\Int}{\code{Int}}
\newcommand{\String}{\code{String}}
\newcommand{\Bool}{\code{Bool}}
\newcommand{\I}{\code{i}}
\newcommand{\J}{\code{j}}


% Rules

% Couleurs
\colorlet{subcolor}{OliveGreen}
\colorlet{targetcolor}{BrickRed}

% Source/elaboration and labels
\newcommand{\rulelabelerecupd}{\rulelabele\text{rec-upd}}


% Presentation
\definecolor{lightyellow}{HTML}{FFFFE0}
\newcommand{\highlight}[1]{\colorbox{GreenYellow}{$#1$}}


% To be retired
\newcommand{\turnsget}{\vdash_{\textrm{get}}}
\newcommand{\turnsput}{\vdash_{\textrm{put}}}
\newcommand{\turnsrec}{\vdash_{\textrm{rec}}}
\newcommand{\rulename}[1]{(\textrm{#1})}



\newcommand{\fand}{{\bf $F_{\&}$}\xspace}

\newcommand{\restrictop}{\setminus}

\newcommand{\orthog}{\perp}

\newcommand{\rulelabelorthog}{\bm{o}}

\newcommand{\rulelabelorthogvar}{\rulelabelorthog\text{var}}
\newcommand{\ruleorthogvar}{
\inferrule* [right=$\rulelabelorthogvar$]
  {\alpha_1 \neq \alpha_2}
  {\alpha_1 \orthog \alpha_2}
}

\newcommand{\rulelabelorthogfun}{\rulelabelorthog\text{fun}}
\newcommand{\ruleorthogfun}{
\inferrule* [right=$\rulelabelorthogfun$]
  {\tau_1 \orthog \tau_3 \\ \tau_2 \orthog \tau_4}
  {\tau_1 \to \tau_2 \orthog \tau_3 \to \tau_4}
}

\newcommand{\rulelabelorthogforall}{\rulelabelorthog\text{forall}}
\newcommand{\ruleorthogforall}{
\inferrule* [right=$\rulelabelorthogforall$]
  {\tau_1 \orthog \subst {\alpha_1} {\alpha_2} \tau_2}
  {\for {\alpha_1} \tau_1 \orthog \for {\alpha_2} \tau_2}
}

\newcommand{\rulelabelorthogandleft}{\rulelabelorthog{\text{and-left}}}
\newcommand{\ruleorthogandleft}{
\inferrule* [right=$\rulelabelorthogandleft$]
  {\tau_1 \orthog \tau_3 \\ \tau_2 \orthog \tau_3}
  {\tau_1 \andop \tau_2 \orthog \tau_3}
}

\newcommand{\rulelabelorthogandright}{\rulelabelorthog{\text{and-right}}}
\newcommand{\ruleorthogandright}{
\inferrule* [right=$\rulelabelorthogandright$]
  {\tau_1 \orthog \tau_2 \\ \tau_1 \orthog \tau_3}
  {\tau_1 \orthog \tau_2 \andop \tau_3}
}

\newcommand{\rulelabelorthogrec}{\rulelabelorthog\text{rec}}
\newcommand{\ruleorthogrec}{
\inferrule* [right=$\rulelabelorthogrec$]
  {l_1 \neq l_2}
  {\recty {l_1} {\tau_1} \orthog \recty {l_2} {\tau_2}}
}
\newcommand{\judgeewf}[2]{#1 \; {\turns} \; #2}

\newcommand{\rulelabelewf}{\bm{wf}}

\newcommand{\rulelabelewfvar}{\rulelabelewf\text{var}}
\newcommand{\rulelabelewftop}{\rulelabelewf\text{top}}
\newcommand{\rulelabelewffun}{\rulelabelewf\text{fun}}
\newcommand{\rulelabelewfforall}{\rulelabelewf\text{forall}}
\newcommand{\rulelabelewfand}{\rulelabelewf{\text{and}}}
\newcommand{\rulelabelewfrec}{\rulelabelewf\text{rec}}

\newcommand{\ruleewf}{
\inferrule* [right=$\rulelabelewf$]
  {\ftv \tau \in \gamma}
  {\judgeewf \gamma \tau}
}

\newcommand{\rulelabeltwf}{\rulelabelt\text{wf}}
\newcommand{\ruletwf}{
\inferrule* [right=$\rulelabeltwf$]
  {\ftv T \in \Gamma}
  {\judgetwf \Gamma T}
}

% Expanded form of well-formedness

\newcommand{\ruleewfvar}{
\inferrule* [right=$\rulelabelewfvar$]
  {\alpha \in \gamma}
  {\judgeewf \gamma \alpha}
}

\newcommand{\ruleewftop}{
\inferrule* [right=$\rulelabelewftop$]
  { }
  {\judgeewf \gamma \top}
}

\newcommand{\ruleewffun}{
\inferrule* [right=$\rulelabelewffun$]
  {\judgeewf \gamma {\tau_1} \\ \judgeewf \Gamma {\tau_2}}
  {\judgeewf \gamma {\tau_1 \to \tau_2}}
}

\newcommand{\ruleewfforall}{
\inferrule* [right=$\rulelabelewfforall$]
  {\judgeewf {\gamma, \alpha} \tau}
  {\judgeewf \gamma {\for \alpha \tau}}
}

\newcommand{\ruleewfand}{
\inferrule* [right=$\rulelabelewfand$]
  {\judgeewf \gamma {\tau_1} \\ \judgeewf \Gamma {\tau_2} \\ \tau_1 \orthog \tau_2}
  {\judgeewf \gamma {\tau_1 \andop \tau_2}}
}

\newcommand{\ruleewfrec}{
\inferrule* [right=$\rulelabelewfrec$]
  {\judgeewf \gamma \tau}
  {\judgeewf \gamma {\recty l \tau}}
}
\newcommand{\rulelabelsub}{\bm{sub}}

\newcommand{\rulelabelsubvar}{\rulelabelsub\text{var}}
\newcommand{\rulesubvar}{
\inferrule* [right=$\rulelabelsubvar$]
  { }
  {\alpha \subtype \alpha}
}

\newcommand{\rulelabelsubtop}{\rulelabelsub\text{top}}
\newcommand{\rulesubtop}{
\inferrule* [right=$\rulelabelsubtop$]
  { }
  {\tau \subtype \top}
}

\newcommand{\rulelabelsubfun}{\rulelabelsub\text{fun}}
\newcommand{\rulesubfun}{
\inferrule* [right=$\rulelabelsubfun$]
  {\tau_3 \subtype \tau_1 \\ \tau_2 \subtype \tau_4}
  {\tau_1 \to \tau_2 \subtype \tau_3 \to \tau_4}
}

\newcommand{\rulelabelsubforall}{\rulelabelsub\text{forall}}
\newcommand{\rulesubforall}{
\inferrule* [right=$\rulelabelsubforall$]
  {\tau_1 \subtype \subst {\alpha_1} {\alpha_2} \tau_2}
  {\for {\alpha_1} \tau_1 \subtype \for {\alpha_2} \tau_2}
}

\newcommand{\rulelabelsuband}{\rulelabelsub\text{and}}
\newcommand{\rulesuband}{
\inferrule* [right=$\rulelabelsuband$]
  {\tau_1 \subtype \tau_2 \\ \tau_1 \subtype \tau_3}
  {\tau_1 \subtype \tau_2 \andop \tau_3}
}

\newcommand{\rulelabelsubandleft}{\rulelabelsub{\text{and}_1}}
\newcommand{\rulesubandleft}{
\inferrule* [right=$\rulelabelsubandleft$]
  {\tau_1 \subtype \tau_3}
  {\tau_1 \andop \tau_2 \subtype \tau_3}
}

\newcommand{\rulelabelsubandright}{\rulelabelsub{\text{and}_2}}
\newcommand{\rulesubandright}{
\inferrule* [right=$\rulelabelsubandright$]
  {\tau_2 \subtype \tau_3}
  {\tau_1 \andop \tau_2 \subtype \tau_3}
}

\newcommand{\rulelabelsubrec}{\rulelabelsub\text{rec}}
\newcommand{\rulesubrec}{
\inferrule* [right=$\rulelabelsubrec$]
  {\tau_1 \subtype \tau_2}
  {\recty l {\tau_1} \subtype \recty l {\tau_2}}
}
\newcommand{\judgeselect}[3]{#1 \bullet #2 = #3}

% select
\newcommand{\rulelabelselect}{\bm{select}}
\newcommand{\ruleget}{
  \inferrule* [right=$\rulelabelselect$]
  { }
  {\judgeselect {\recty l \tau} l \tau}
}

% select1
\newcommand{\rulelabelselectleft}{{\rulelabelselect}_1}
\newcommand{\rulegetleft}{
  \inferrule* [right=$\rulelabelselectleft$]
  {\judgeselect {\tau_1} l \tau}
  {\judgeselect {\tau_1 \andop \tau_2} l \tau}
}

% select2
\newcommand{\rulelabelselectright}{{\rulelabelselect}_2}
\newcommand{\rulegetright}{
  \inferrule* [right=$\rulelabelselectright$]
  {\judgeselect {\tau_2} l \tau}
  {\judgeselect {\tau_1 \andop \tau_2} l \tau}
}

\newcommand{\judgerestrict}[3]{#1 \bm{\restrictop} #2 = #3}

% restrict
\newcommand{\rulelabelrestrict}{\bm{restrict}}
\newcommand{\rulerestrict}{
  \inferrule* [right=$\rulelabelrestrict$]
  { }
  {\judgerestrict {\recty l \tau} l \top}
}

% restrict1
\newcommand{\rulelabelrestrictleft}{{\rulelabelrestrict}_1}
\newcommand{\rulerestrictleft}{
  \inferrule* [right=$\rulelabelrestrictleft$]
  {\judgerestrict {\tau_1} l \tau}
  {\judgerestrict {\tau_1 \andop \tau_2} l {\tau \andop \tau_2}}
}

% restrict2
\newcommand{\rulelabelrestrictright}{{\rulelabelrestrict}_2}
\newcommand{\rulerestrictright}{
  \inferrule* [right=$\rulelabelrestrictright$]
  {\judgerestrict {\tau_2} l \tau}
  {\judgerestrict {\tau_1 \andop \tau_2} l {\tau_1 \andop \tau}}
}

%%%%%%%%%%%%%%%%%%%%%%%%%%%%%%%%%%%%%%%%%%%%%%%%%%%%%%%%%%%%%%%%%%%%%%%%
% Typing
%%%%%%%%%%%%%%%%%%%%%%%%%%%%%%%%%%%%%%%%%%%%%%%%%%%%%%%%%%%%%%%%%%%%%%%%

\newcommand{\judgee}[3]{#1 \; \turns \; #2 \; : \; #3}
\newcommand{\rulelabele}{\bm{ty}}

% var
\newcommand{\rulelabelevar}{\rulelabele\text{var}}
\newcommand{\ruleevar} {
\inferrule* [right=$\rulelabelevar$]
  {(x,\tau) \in \gamma}
  {\judgee \gamma x \tau}
}

% top
\newcommand{\rulelabeletop}{\rulelabele\text{top}}
\newcommand{\ruleetop} {
\inferrule* [right=$\rulelabeletop$]
  { }
  {\judgee \gamma \top \top}
}

% lam
\newcommand{\rulelabelelam}{\rulelabele\text{lam}}
\newcommand{\ruleelam} {
\inferrule* [right=$\rulelabelelam$]
  {\judgee {\gamma, x \hast \tau} e {\tau_1} \\ \judgeewf \gamma \tau}
  {\judgee \gamma {\lam x \tau e} {\tau \to \tau_1}}
}

% app
\newcommand{\rulelabeleapp}{\rulelabele\text{app}}
\newcommand{\ruleeapp}{
\inferrule* [right=$\rulelabeleapp$]
  {\judgee \gamma {e_1} {\tau_1 \to \tau_2} \\
   \judgee \gamma {e_2} {\tau_3} \\
   \tau_3 \subtype \tau_1}
  {\judgee \gamma {\app {e_1} {e_2}} {\tau_2}}
}

% blam
\newcommand{\rulelabeleblam}{\rulelabele\text{blam}}
\newcommand{\ruleeblam}{
\inferrule* [right=$\rulelabeleblam$]
  {\judgee {\gamma, \alpha} e \tau}
  {\judgee \gamma {\blam \alpha e} {\for \alpha \tau}}
}

% tapp
\newcommand{\rulelabeletapp}{\rulelabele\text{tapp}}
\newcommand{\ruleetapp}{
\inferrule* [right=$\rulelabeletapp$]
  {\judgee \gamma e {\for \alpha {\tau_1}} \\ \judgeewf \gamma \tau}
  {\judgee \gamma {\tapp e \tau} {\subst \tau \alpha \tau_1}}
}

% merge
\newcommand{\rulelabelemerge}{\rulelabele\text{merge}}
\newcommand{\ruleemerge}{
\inferrule* [right=$\rulelabelemerge$]
  {\judgee \gamma {e_1} {\tau_1} \\
   \judgee \gamma {e_2} {\tau_2}}
  {\judgee \gamma {e_1 \mergeop e_2} {\tau_1 \andop \tau_2}}
}

% rec-construct
\newcommand{\rulelabelerecconstruct}{\rulelabele\text{rec-construct}}
\newcommand{\ruleerecconstruct}{
\inferrule* [right=$\rulelabelerecconstruct$]
  {\judgee \gamma e \tau}
  {\judgee \gamma {\reccon l e} {\recty l \tau}}
}

% rec-select
\newcommand{\rulelabelerecselect}{\rulelabele\text{rec-select}}
\newcommand{\ruleerecselect}{
\inferrule* [right=$\rulelabelerecselect$]
  {\judgee \gamma e \tau \\
   \judgeselect \tau l {\tau_1}}
  {\judgee \gamma {e.l} {\tau_1}}
}

% rec-restrict
\newcommand{\rulelabelerecrestrict}{\rulelabele\text{rec-restrict}}
\newcommand{\ruleerecrestrict}{
\inferrule* [right=$\rulelabelerecrestrict$]
  {\judgee \gamma e \tau \\
   \judgerestrict \tau l {\tau_1}}
  {\judgee \gamma {e \restrictop l} {\tau_1}}
}


\newcommand{\kwsig}{\keyword{sig}}
\newcommand{\kwextends}{\keyword{extends}}
\newcommand{\kwalgebra}{\keyword{algebra}}
\newcommand{\kwimplements}{\keyword{implements}}
\newcommand{\kwdata}{\keyword{data}}
\newcommand{\kwfrom}{\keyword{from}}
\newcommand{\trans}[4]{#1 \; \turns \; #2 \; : \; #3 \; \Rightarrow \; #4}
\newcommand{\Tau}{\mathrm{T}}

\newcommand{\sig}[5]{
  \kwsig \; #1 {[} \overline{#2} {]} \;
  \kwwhere \; \overline{#3 : #4} \; \kwin \; #5
}
\newcommand{\sigext}[7]{
  \kwsig \; #1 {[} \overline{#2} {]} \;
  \kwextends \; \overline{#3 {[} \overline{#4} {]}} \;
  \kwwhere \; \overline{#5 : #6} \; \kwin \; #7
}

\newcommand{\alg}[8]{
  \kwalgebra \; #1 \; \kwimplements \; \overline{#2 {[} \overline{#3} {]}} \;
  \kwwhere \; \overline{#4 {@(} #5 \; \overline{#6} {)=} \; #7} \; \kwin \; #8
}

\newcommand{\algext}[9]{
  \kwalgebra \; #1 \; \kwextends \; \overline{#2} \;
  \kwimplements \; \overline{#3 {[} \overline{#4} {]}} \;
  \kwwhere \; \overline{#5 {@(} #6 \; \overline{#7} {)=} \; #8} \; \kwin \; #9
}

\newcommand{\data}[5]{
  \kwdata \; #1 \; \kwfrom \; #2 {[} \overline{#3} {]} . #4 \;
  \kwin \; #5
}

\newcommand{\rulesig}{
\inferrule* [right=]
  {\trans {\Gamma, s {[} \overline{\alpha} {]}\to\overline{l:\tau}} e {\tau_*} E }
  {\trans \Gamma {\sig s \alpha l \tau e} {\tau_*} {\letexpr {merge_s : ...} {...} E}}
}

\newcommand{\rulesigext}{
\inferrule* [right=]
  {\overline{\judgeewf \Gamma {s_2[\overline{\alpha_2}]}} \\ \trans {\Gamma, s_1[\overline{\alpha_1}]\to\textbf{U}_\varnothing\overline{[\overline{\alpha}/\overline{\alpha_2}]\Gamma(s_2)} \; \textbf{U}_\leftarrow\overline{l:\tau}} e {\tau_*} E }
  {\trans \Gamma {\sigext {s_1} {\alpha_1} {s_2} \alpha l \tau e} {\tau_*} {\letexpr {merge_{s_1} : ...} {...} E}}
}

\newcommand{\rulealg}{
\inferrule* [right=]
  {\overline{\judgeewf \Gamma {s[\overline{\alpha}]}} \\ \overline{\trans {\Gamma, \overline{x_1} : [\overline{\tau}/\overline{\alpha}]gen2_A(l_1)} {e_1} {\tau_1} {E_1}} \\ \trans {\Gamma, x : \andop\overline{[\overline{\tau}/\overline{\alpha}]\Gamma(s)}, x \multimap \overline{s[\overline{\tau}]}} e {\tau_*} E}
  {\trans \Gamma {\alg x s \tau l {l_1} {x_1} {e_1} e} {\tau_*} {\\ \letexpr {x : \red{\andop\overline{[\overline{\tau}/\overline{\alpha}]\Gamma(s)}}} {\red{\{\overline{l_1 = \lam {\overline{x_1}} {[\overline{\tau}/\overline{\alpha}]gen2_A(l_1)} {\{l = E_1\}}}\}}} E}}
}

\newcommand{\rulealgext}{
\inferrule* [right=]
  {\overline{\judgeewf \Gamma {s[\overline{\alpha}]}} \\ \overline{\judgeewf \Gamma {x_0}} \\ \overline{\trans {\Gamma, \overline{x_1} : [\overline{\tau}/\overline{\alpha}]gen2_A(l_1)} {e_1} {\tau_1} {E_1}} \\ \trans {\Gamma, x : \andop\overline{[\overline{\tau}/\overline{\alpha}]\Gamma(s)}, x \multimap \overline{s[\overline{\tau}]}} e {\tau_*} E}
  {\trans \Gamma {\algext x {x_0} s \tau l {l_1} {x_1} {e_1} e} {\tau_*} {\\ \letexpr {\red{x : \andop\overline{[\overline{\tau}/\overline{\alpha}]\Gamma(s)}}} {\red{\overline{x_0} \mergeop \{\overline{l_1 = \lam {\overline{x_1}} {[\overline{\tau}/\overline{\alpha}]gen2_A(l_1)} {\{l = E_1\}}}\}}} E}}
}

%\newcommand{\ruledatatype}{
%\inferrule* [right=]
%  {\judgeewf \Gamma {s[\overline{\alpha}]} \to \overline{l : \tau} \\ \trans \Gamma {[[\overline{\tau}/(\overline{\alpha_0}\backslash\alpha_1)]\{accept : \for {\alpha_1} {s[\overline{\alpha_0}] \to \alpha_1}\}/d[\overline{\tau}]]e} {\tau_*} E}
%  {\trans \Gamma {\data d s {\alpha_0} {\alpha_1} e} {\tau_*} {\letexpr {\overline{gen3(l)}} {...} E}}
%}

\newcommand{\ruledatatype}{
\inferrule* [right=\red{need tcheck}]
  {\judgeewf \Gamma {s[\overline{\alpha}]} \to \overline{l : \tau} \\ \trans {\Gamma, d \rightsquigarrow s[\overline{\alpha_0}].\alpha_1 : \{accept : \for {\alpha_1} {s[\overline{\alpha_0}] \to \alpha_1}\}} e {\tau_*} E}
  {\trans \Gamma {\data d s {\alpha_0} {\alpha_1} e} {\tau_*} {\letexpr {\overline{gen3(l)}} {...} E}}
}

\newcommand{\rulebuilda}{
\inferrule* [right=\red{need tcheck}]
  {...}
  {\trans \Gamma {\letexpr {x : d[\overline{\tau}]} {e_1} e} {\tau_*} {...}}
}

\newcommand{\rulebuildb}{
\inferrule* [right=\red{need tcheck}]
  {\judgeewf \Gamma d \\ \trans {\Gamma, \overline{x_1} : \overline{\tau_1}, \overline{x_2} : d[\overline{\tau}]} {e_1} {d[\overline{\tau}]} {E_1} \\ \trans {\Gamma, x : \overline{\tau_1} \to \overline{d[\overline{\tau}]} \to d[\overline{\tau}]} {e} {\tau_*} {E}}
  {\trans \Gamma {\letexpr {x \; (\overline{x_1} : \overline{\tau_1}) \; (\overline{x_2} : d[\overline{\tau}]) : d[\overline{\tau}]} {e_1} e} {\tau_*} {\letexpr x {[gen4(d)]E_1} E}}
}

\newcommand{\ruleinsta}{
\inferrule* [right=]
  {\overline{\judgeewf \Gamma {x \multimap s[\overline{\tau}]}} \\ \judgeewf \Gamma {s[\overline{\alpha}]}}
  {\trans \Gamma {\texttt{<}\overline{x}\texttt{>}} {s[\andop\overline{\tau}]} {merge_s\overline{[\overline{\tau}]} \; \overline{x}}}
}

\newcommand{\mergealg}{
merge_s : \for {\overline{\alpha_A}} {\for {\overline{\alpha_B}} {s[\overline{\alpha_A}] \to s[\overline{\alpha_B}] \to s[\overline{\alpha_A\andop\alpha_B}]}} = \blam {\overline{\alpha_A}} {\blam {\overline{\alpha_B}} {\lam {alg_1} {s[\overline{\alpha_A}]} {\lam {alg_2} {s[\overline{\alpha_B}]} {\{\overline{l=[\overline{\alpha_A\andop\alpha_B} / \overline{\alpha}] gen(l)}\}}}}}
}

\newcommand{\ruleexpandsig} {
\inferrule* [right=]
    {\judgeewf \Gamma {s[\overline{\alpha}]\to\tau_*}}
    {\judgeewf \Gamma {s[\overline{\tau_0}]\Rightarrow[\overline{\tau_0}/\overline{\alpha}]\tau_*}}
}

\newcommand{\ruleexpanddata} {
\inferrule* [right=]
    {\judgeewf \Gamma {d \rightsquigarrow s[\overline{\alpha_0}].\alpha_1 : \tau_*} \\ \judgeewf \Gamma {s[\overline{\alpha}]\to\overline{l:\tau}}}
    {\judgeewf \Gamma {d[\overline{\tau_0}]\Rightarrow[\overline{\tau_0}/(\overline{\alpha_0}\backslash\alpha_1)]\tau_*}}
}


\begin{document}

\section{Syntax}

\subsection{Source Syntax}

\begin{displaymath}
    \begin{array}{l}
      \begin{array}{llrl}
        \text{Types}
        & T & \Coloneqq & \alpha \mid \top \mid \tau_1 \to \tau_2 \mid \for \alpha \tau \mid \tau_1 \andop \tau_2 \mid \recty l \tau \\
        \text{Expressions}
        & E & \Coloneqq & x \mid \top \mid \lam x \tau e \mid \app {e_1} {e_2} \mid \blam \alpha e \mid \tapp e \tau \mid e_1 \mergeop e_2 \mid \reccon l e \mid e.l \mid e \restrictop l \\
        &   & \mid & \sig s \alpha l \tau e \\
        &   & \mid & \sigext {s_1} {\alpha_1} {s_2} \alpha l \tau e \\
        &   & \mid & \alg x s \tau l {l_1} {x_1} {e_1} e \\
        &   & \mid & \algext x {x_0} s \tau l {l_1} {x_1} {e_1} e \\
        &   & \mid & \data d s {\alpha_0} {\alpha_1} e \\
        %&   & \mid & \letexpr {x : d[\overline{\tau}]} {e_1} e \\
        &   & \mid & \letexpr {x \; (\overline{x_1} : \overline{\tau_1}) \; (\overline{x_2} : d[\overline{\tau}]) : d[\overline{\tau}]} {e_1} e \\
        &   & \mid & e[\overline{\tau}]\texttt{<}\overline{x}\texttt{>} \\
        \text{Contexts} & \Gamma & \Coloneqq & \epsilon \mid \Gamma, \alpha \mid \Gamma, x \hast \tau \\
        &   & \mid & \Gamma, s {[} \overline{\alpha} {]}\to\overline{l:\tau} \\
        &   & \mid & \Gamma, x \multimap \overline{s[\overline{\tau}]} \\
        &   & \mid & \Gamma, d \rightsquigarrow s[\overline{\alpha_0}].\alpha_1 : \tau \\
        \text{Labels} & l &  & \text{(fields)} \\
        & s &  & \text{(interfaces)} \\
        & d &  & \text{(datatypes)}\\
        \text{Syntactic sugars} & \circ & \Coloneqq & s[\overline{\alpha_0}] \\
        & \bullet & \Coloneqq & [\overline{\alpha_0}/\overline{\alpha}]\Gamma(s) \\
        & \circ & \Coloneqq & d(\overline{\tau_0}) \\
        & \bullet & \Coloneqq & [\overline{\tau_0}/(\overline{\alpha_0}\backslash\alpha_1)]\Gamma(d)
      \end{array}
    \end{array}
\end{displaymath}

\subsection{Target Syntax}

\begin{displaymath}
    \begin{array}{l}
      \begin{array}{llrl}
        \text{Types}
        & T & \Coloneqq & \alpha \mid \top \mid \tau_1 \to \tau_2 \mid \for \alpha \tau \mid \tau_1 \andop \tau_2 \mid \recty l \tau \\
        \text{Expressions}
        & E & \Coloneqq & x \mid \top \mid \lam x \tau e \mid \app {e_1} {e_2} \mid \blam \alpha e \mid \tapp e \tau \mid e_1 \mergeop e_2 \mid \reccon l e \mid e.l \mid e \restrictop l\\
        \text{Contexts} & \Gamma & \Coloneqq & \epsilon \mid \Gamma, \alpha \mid \Gamma, x \hast \tau \\
        \text{Labels} & l \\
        \text{Syntactic sugars} & \circ & \Coloneqq & \letexpr {x:\tau} {e_1} {e_2} \\
        & \bullet & \Coloneqq & \app {(\lam x \tau {e_2})} {e_1}
      \end{array}
    \end{array}
\end{displaymath}

\section{Translation Rules}

\begin{mathpar}
    \framebox{$ \judgeewf \Gamma {e:\tau\Rightarrow E} $}

    \rulesig

    \rulesigext

    \rulealg

    \rulealgext

    \ruledatatype

    %\rulebuilda

    \rulebuildb

    \ruleinsta
\end{mathpar}

~\\

$merge_s$: the merge algebra for object algebra interface $s$.\\

$\mergealg$\\

$gen(l)$: $\lam {\overline{x}} {...} {alg1.\overline{x} \mergeop alg2.\overline{x}}$.\\

$gen2(l)$: get the type from context $\Gamma(s).l$.\\

$gen3(l)$: for each case $l$, generate an auxiliary function for building structures. Only consider those with return type $\alpha_1$ in $\data d s {\alpha_0} {\alpha_1} e$.\\

$[gen4(d)]$: $[\overline{l[\overline{\tau}]}/\overline{l}]$. Only when $d \rightsquigarrow s[\overline{\alpha_0}].\alpha_1$, $l$ is a constructor in $s$, and $gen3(l)$ exists.

\section{Auxiliary Rules for Expanding Types}

\section{Amendment}

\subsection{In translation: alg}

Question: Type-check for $\tau_1$ <: the return type in $\Gamma(s)$?

\subsection{In translation: algext}

Question: Type-check for $\overline{s[\overline{\tau}]}$ = $\overline{\tau_0}$ + ...?

\subsection{In translation: datatype}

Question: Check if $\alpha_1\in\overline{\alpha_0}$? Not sure if something like ... makes sense.

\subsection{In translation: insta}

Question: Check if $e$ has the field ``accept''? And the relationship between $\tau_0$ and $\tau_*$?

\subsection{Critical: algebras in context?}

Question: Put algebras into context? Need to check more for types in that case. But instantiation becomes more concise. Currently the translation rule for instantiation doesn't really work (it doesn't know which interface to use, since merge algebras are namespaced). The merge algebra is limited. And some constructors potentially cannot be generated automatically.

\subsection{Extension: sig as type? structure building?}

Question: Currently a structure can only be built from datatype. With signatures as types, the code could be more flexible.

\subsection{Notes}

The rules should support both special syntax for algebras and common syntax.
\begin{itemize}
\item \textbf{sig:} (1) in the environment; (2) as a type synonym.
\item \textbf{alg:} (1) in the environment; (2) as a function.
\item \textbf{data:} (1) in the environment; (2) as a type synonym.
\end{itemize}

\end{document}
